\documentclass[12pt]{article}
\usepackage[portuguese]{babel}
\usepackage[a4paper, left=30mm, right=20mm, top=30mm, bottom=20mm]{geometry}
\usepackage{setspace}
\usepackage{graphicx}
\usepackage{indentfirst}
\usepackage{amsmath}
\usepackage{siunitx}
\usepackage{csvsimple}
\usepackage{float}
\usepackage{booktabs}
\usepackage{hyperref}

% Parâmetros da Capa
\newcommand{\titulo}{2ª Tarefa de Métodos Computacionais para Equações Diferenciais}
\newcommand{\autor}{Guilherme Cesar Tomiasi}
\newcommand{\cidade}{Presidente Prudente}
% ================================================================================

\begin{document}
%Definir o espaçamento entre as linhas para 1.5
\setstretch{1.5}

%Capa------------------------------------------------------
\thispagestyle{empty}

\begin{minipage}[c]{0.3\textwidth}
\includegraphics[width=\textwidth]{../images/unesp.png}
\end{minipage}
\hspace{10pt}
\begin{minipage}[c]{0.6\textwidth}
\textbf{\uppercase{Universidade Estadual Paulista \\``Júlio de Mesquita Filho"}}
\end{minipage}

\setstretch{1.0}
\begin{center}
    \noindent\hrulefill

    \textbf{FCT - Faculdade de Ciências e Tecnologia}

    \textbf{DMC - Departamento de Matemática e Computação}

    \textbf{Pós-Graduação em Matemática Aplicada e Computacional}
\end{center}

\vspace*{\fill}

\setstretch{1.5}
\begin{center}
    \titulo\\
    \autor
\end{center}

\vspace{1cm}

\vspace*{\fill}

\begin{center}
    \uppercase{\cidade\\\today}
\end{center}
\singlespacing
\newpage
\tableofcontents

\newpage
\section{Descrição da Tarefa}
O objetivo dessa Tarefa é explicitar a diferença no comportamento de métodos numéricos para a solução de PVIs (isto é, Problemas de Valor Inicial) na solução de Equações Diferenciais Ordinárias. São realizados 3 exercícios, testando os seguintes métodos:
\begin{itemize}
    \item Euler Explícito, método de \textbf{Primeira Ordem};
    \item Euler Implícito, método de \textbf{Primeira Ordem};
    \item Método dos Trapézios, método de \textbf{Segunda Ordem};
    \item Método Runge-Kutta Clássico de \textbf{Terceira Ordem};
    \item Método Runge-Kutta Clássico de \textbf{Quarta Ordem};
    \item Método de Euler Modificado de \textbf{Segunda Ordem};
    \item Método de Euler Aperfeiçoado de \textbf{Segunda Ordem};
\end{itemize}
Para cada um dos métodos, são aplicadas quatro tamanhos diferentes de malha computacional, sendo esses tamanhos:
\begin{enumerate}
    \item $h=0.1$;
    \item $h=0.01$;
    \item $h=0.005$;
    \item $h=0.001$;
\end{enumerate}

Foi realizada uma função que aplica todos os métodos explícitos, e um método específico foi preparado para a resolução via Euler Implícito. Todos os problemas de Valor Inicial foram descritos em um arquivo \textbf{.JSON}, cujo processo de leitura e de \textit{parsing} foram feitos de forma automática por um \textit{script} na Linguagem Julia. Os resultados para cada malha são salvos em arquivos tabulados (CSV), e as representações via gráficos são salvas em arquivos de imagem (PNG).

\section{Resultados}
Como as tabelas possuem números muito longos em precisão, as mesmas estão muito pequenas. No entanto, o \textit{zoom} pode ser utilizado para permitir a leitura.

\subsection{Exercício 1, item $a$}\subsubsection{$h=0.1$}

\begin{table}[H]
    \centering
    \resizebox{\textwidth}{!}{
\begin{tabular}{ccccc}
\toprule
x & Valor real & Valor (Euler Explícito) & Valor (Euler Implícito) & Valor (Método dos Trapézios)\\
\midrule
0.0000 & 1.0000 & 1.0000 & 1.0000 & 1.0000\\
0.1000 & 0.9097 & 0.9000 & 0.9182 & 0.9100\\
0.2000 & 0.8375 & 0.8200 & 0.8529 & 0.8381\\
0.3000 & 0.7816 & 0.7580 & 0.8026 & 0.7824\\
0.4000 & 0.7406 & 0.7122 & 0.7660 & 0.7416\\
0.5000 & 0.7131 & 0.6810 & 0.7418 & 0.7142\\
0.6000 & 0.6976 & 0.6629 & 0.7289 & 0.6988\\
0.7000 & 0.6932 & 0.6566 & 0.7263 & 0.6944\\
0.8000 & 0.6987 & 0.6609 & 0.7330 & 0.7000\\
0.9000 & 0.7131 & 0.6748 & 0.7482 & 0.7145\\
1.0000 & 0.7358 & 0.6974 & 0.7711 & 0.7371\\
\bottomrule
\end{tabular}
}
    \caption{Valores obtidos para o item $a$ com $h=0.1$}
\end{table}

\begin{table}[H]
    \centering
    \resizebox{\textwidth}{!}{
\begin{tabular}{ccccccc}
\toprule
x & Erro absoluto (Euler Explícito) & Erro relativo (Euler Explícito) & Erro absoluto (Euler Implícito) & Erro relativo (Euler Implícito) & Erro absoluto (Método dos Trapézios) & Erro relativo (Método dos Trapézios)\\
\midrule
0.00e+00 & 0.00e+00 & 0.00e+00 & 0.00e+00 & 0.00e+00 & 0.00e+00 & 0.00e+00\\
1.00e-01 & 9.67e-03 & 1.06e-02 & 8.51e-03 & 9.35e-03 & 3.25e-04 & 3.57e-04\\
2.00e-01 & 1.75e-02 & 2.09e-02 & 1.54e-02 & 1.84e-02 & 5.88e-04 & 7.03e-04\\
3.00e-01 & 2.36e-02 & 3.02e-02 & 2.10e-02 & 2.69e-02 & 7.99e-04 & 1.02e-03\\
4.00e-01 & 2.84e-02 & 3.84e-02 & 2.54e-02 & 3.43e-02 & 9.64e-04 & 1.30e-03\\
5.00e-01 & 3.21e-02 & 4.50e-02 & 2.88e-02 & 4.04e-02 & 1.09e-03 & 1.53e-03\\
6.00e-01 & 3.47e-02 & 4.98e-02 & 3.13e-02 & 4.49e-02 & 1.18e-03 & 1.70e-03\\
7.00e-01 & 3.66e-02 & 5.28e-02 & 3.31e-02 & 4.78e-02 & 1.25e-03 & 1.80e-03\\
8.00e-01 & 3.77e-02 & 5.40e-02 & 3.44e-02 & 4.92e-02 & 1.29e-03 & 1.85e-03\\
9.00e-01 & 3.83e-02 & 5.37e-02 & 3.51e-02 & 4.92e-02 & 1.32e-03 & 1.85e-03\\
1.00e+00 & 3.84e-02 & 5.22e-02 & 3.53e-02 & 4.80e-02 & 1.32e-03 & 1.80e-03\\
\bottomrule
\end{tabular}
}
    \caption{Erros obtidos para o item $a$ com $h=0.1$}
\end{table}

\begin{figure}[H]
    \includegraphics[width=\linewidth]{results/ex1/a/h_0.1.png}
    \caption{Gráfico plotado para o item $a$ quando $h=0.1$}
\end{figure}

\begin{figure}[H]
    \includegraphics[width=\linewidth]{results/ex1/a/h_0.1_abs_error.png}
    \caption{Gráfico plotado para o erro absoluto do item $a$ quando $h=0.1$}
\end{figure}

\begin{figure}[H]
    \includegraphics[width=\linewidth]{results/ex1/a/h_0.1_rel_error.png}
    \caption{Gráfico plotado para o erro relativo do item $a$ quando $h=0.1$}
\end{figure}
\subsubsection{$h=0.01$}

\begin{table}[H]
    \centering
    \resizebox{\textwidth}{!}{
\begin{tabular}{ccccc}
\toprule
x & Valor real & Valor (Euler Explícito) & Valor (Euler Implícito) & Valor (Método dos Trapézios)\\
\midrule
0.0000 & 1.0000 & 1.0000 & 1.0000 & 1.0000\\
0.1000 & 0.9097 & 0.9088 & 0.9106 & 0.9097\\
0.2000 & 0.8375 & 0.8358 & 0.8391 & 0.8375\\
0.3000 & 0.7816 & 0.7794 & 0.7838 & 0.7816\\
0.4000 & 0.7406 & 0.7379 & 0.7433 & 0.7406\\
0.5000 & 0.7131 & 0.7100 & 0.7161 & 0.7131\\
0.6000 & 0.6976 & 0.6943 & 0.7009 & 0.6976\\
0.7000 & 0.6932 & 0.6897 & 0.6966 & 0.6932\\
0.8000 & 0.6987 & 0.6950 & 0.7022 & 0.6987\\
0.9000 & 0.7131 & 0.7095 & 0.7168 & 0.7132\\
1.0000 & 0.7358 & 0.7321 & 0.7394 & 0.7358\\
\bottomrule
\end{tabular}
}
    \caption{Valores obtidos para o item $a$ com $h=0.01$}
\end{table}

\begin{table}[H]
    \centering
    \resizebox{\textwidth}{!}{
\begin{tabular}{ccccccc}
\toprule
x & Erro absoluto (Euler Explícito) & Erro relativo (Euler Explícito) & Erro absoluto (Euler Implícito) & Erro relativo (Euler Implícito) & Erro absoluto (Método dos Trapézios) & Erro relativo (Método dos Trapézios)\\
\midrule
0.00e+00 & 0.00e+00 & 0.00e+00 & 0.00e+00 & 0.00e+00 & 0.00e+00 & 0.00e+00\\
1.00e-01 & 9.11e-04 & 1.00e-03 & 8.99e-04 & 9.88e-04 & 3.04e-06 & 3.34e-06\\
2.00e-01 & 1.65e-03 & 1.97e-03 & 1.63e-03 & 1.94e-03 & 5.50e-06 & 6.57e-06\\
3.00e-01 & 2.24e-03 & 2.86e-03 & 2.21e-03 & 2.83e-03 & 7.46e-06 & 9.55e-06\\
4.00e-01 & 2.70e-03 & 3.64e-03 & 2.67e-03 & 3.60e-03 & 9.00e-06 & 1.22e-05\\
5.00e-01 & 3.05e-03 & 4.28e-03 & 3.02e-03 & 4.23e-03 & 1.02e-05 & 1.43e-05\\
6.00e-01 & 3.31e-03 & 4.74e-03 & 3.28e-03 & 4.70e-03 & 1.11e-05 & 1.59e-05\\
7.00e-01 & 3.49e-03 & 5.04e-03 & 3.46e-03 & 4.99e-03 & 1.17e-05 & 1.68e-05\\
8.00e-01 & 3.61e-03 & 5.17e-03 & 3.58e-03 & 5.12e-03 & 1.21e-05 & 1.73e-05\\
9.00e-01 & 3.68e-03 & 5.15e-03 & 3.64e-03 & 5.11e-03 & 1.23e-05 & 1.72e-05\\
1.00e+00 & 3.69e-03 & 5.02e-03 & 3.66e-03 & 4.98e-03 & 1.24e-05 & 1.68e-05\\
\bottomrule
\end{tabular}
}
    \caption{Erros obtidos para o item $a$ com $h=0.01$}
\end{table}

\begin{figure}[H]
    \includegraphics[width=\linewidth]{results/ex1/a/h_0.01.png}
    \caption{Gráfico plotado para o item $a$ quando $h=0.01$}
\end{figure}

\begin{figure}[H]
    \includegraphics[width=\linewidth]{results/ex1/a/h_0.01_abs_error.png}
    \caption{Gráfico plotado para o erro absoluto do item $a$ quando $h=0.01$}
\end{figure}

\begin{figure}[H]
    \includegraphics[width=\linewidth]{results/ex1/a/h_0.01_rel_error.png}
    \caption{Gráfico plotado para o erro relativo do item $a$ quando $h=0.01$}
\end{figure}
\subsubsection{$h=0.005$}

\begin{table}[H]
    \centering
    \resizebox{\textwidth}{!}{
\begin{tabular}{ccccc}
\toprule
x & Valor real & Valor (Euler Explícito) & Valor (Euler Implícito) & Valor (Método dos Trapézios)\\
\midrule
0.0000 & 1.0000 & 1.0000 & 1.0000 & 1.0000\\
0.1000 & 0.9097 & 0.9092 & 0.9101 & 0.9097\\
0.2000 & 0.8375 & 0.8366 & 0.8383 & 0.8375\\
0.3000 & 0.7816 & 0.7805 & 0.7827 & 0.7816\\
0.4000 & 0.7406 & 0.7393 & 0.7420 & 0.7406\\
0.5000 & 0.7131 & 0.7115 & 0.7146 & 0.7131\\
0.6000 & 0.6976 & 0.6960 & 0.6993 & 0.6976\\
0.7000 & 0.6932 & 0.6914 & 0.6949 & 0.6932\\
0.8000 & 0.6987 & 0.6969 & 0.7005 & 0.6987\\
0.9000 & 0.7131 & 0.7113 & 0.7150 & 0.7131\\
1.0000 & 0.7358 & 0.7339 & 0.7376 & 0.7358\\
\bottomrule
\end{tabular}
}
    \caption{Valores obtidos para o item $a$ com $h=0.005$}
\end{table}

\begin{table}[H]
    \centering
    \resizebox{\textwidth}{!}{
\begin{tabular}{ccccccc}
\toprule
x & Erro absoluto (Euler Explícito) & Erro relativo (Euler Explícito) & Erro absoluto (Euler Implícito) & Erro relativo (Euler Implícito) & Erro absoluto (Método dos Trapézios) & Erro relativo (Método dos Trapézios)\\
\midrule
0.00e+00 & 0.00e+00 & 0.00e+00 & 0.00e+00 & 0.00e+00 & 0.00e+00 & 0.00e+00\\
1.00e-01 & 4.54e-04 & 4.99e-04 & 4.51e-04 & 4.96e-04 & 7.57e-07 & 8.32e-07\\
2.00e-01 & 8.21e-04 & 9.81e-04 & 8.16e-04 & 9.75e-04 & 1.37e-06 & 1.64e-06\\
3.00e-01 & 1.11e-03 & 1.43e-03 & 1.11e-03 & 1.42e-03 & 1.86e-06 & 2.38e-06\\
4.00e-01 & 1.34e-03 & 1.82e-03 & 1.34e-03 & 1.80e-03 & 2.24e-06 & 3.03e-06\\
5.00e-01 & 1.52e-03 & 2.13e-03 & 1.51e-03 & 2.12e-03 & 2.54e-06 & 3.56e-06\\
6.00e-01 & 1.65e-03 & 2.37e-03 & 1.64e-03 & 2.35e-03 & 2.75e-06 & 3.95e-06\\
7.00e-01 & 1.74e-03 & 2.51e-03 & 1.73e-03 & 2.50e-03 & 2.91e-06 & 4.19e-06\\
8.00e-01 & 1.80e-03 & 2.58e-03 & 1.79e-03 & 2.57e-03 & 3.01e-06 & 4.30e-06\\
9.00e-01 & 1.83e-03 & 2.57e-03 & 1.83e-03 & 2.56e-03 & 3.06e-06 & 4.29e-06\\
1.00e+00 & 1.84e-03 & 2.51e-03 & 1.84e-03 & 2.49e-03 & 3.08e-06 & 4.18e-06\\
\bottomrule
\end{tabular}
}
    \caption{Erros obtidos para o item $a$ com $h=0.005$}
\end{table}

\begin{figure}[H]
    \includegraphics[width=\linewidth]{results/ex1/a/h_0.005.png}
    \caption{Gráfico plotado para o item $a$ quando $h=0.005$}
\end{figure}

\begin{figure}[H]
    \includegraphics[width=\linewidth]{results/ex1/a/h_0.005_abs_error.png}
    \caption{Gráfico plotado para o erro absoluto do item $a$ quando $h=0.005$}
\end{figure}

\begin{figure}[H]
    \includegraphics[width=\linewidth]{results/ex1/a/h_0.005_rel_error.png}
    \caption{Gráfico plotado para o erro relativo do item $a$ quando $h=0.005$}
\end{figure}
\subsubsection{$h=0.001$}

\begin{table}[H]
    \centering
    \resizebox{\textwidth}{!}{
\begin{tabular}{ccccc}
\toprule
x & Valor real & Valor (Euler Explícito) & Valor (Euler Implícito) & Valor (Método dos Trapézios)\\
\midrule
0.0000 & 1.0000 & 1.0000 & 1.0000 & 1.0000\\
0.1000 & 0.9097 & 0.9096 & 0.9098 & 0.9097\\
0.2000 & 0.8375 & 0.8373 & 0.8376 & 0.8375\\
0.3000 & 0.7816 & 0.7814 & 0.7819 & 0.7816\\
0.4000 & 0.7406 & 0.7404 & 0.7409 & 0.7406\\
0.5000 & 0.7131 & 0.7128 & 0.7134 & 0.7131\\
0.6000 & 0.6976 & 0.6973 & 0.6980 & 0.6976\\
0.7000 & 0.6932 & 0.6928 & 0.6935 & 0.6932\\
0.8000 & 0.6987 & 0.6983 & 0.6990 & 0.6987\\
0.9000 & 0.7131 & 0.7128 & 0.7135 & 0.7131\\
1.0000 & 0.7358 & 0.7354 & 0.7361 & 0.7358\\
\bottomrule
\end{tabular}
}
    \caption{Valores obtidos para o item $a$ com $h=0.001$}
\end{table}

\begin{table}[H]
    \centering
    \resizebox{\textwidth}{!}{
\begin{tabular}{ccccccc}
\toprule
x & Erro absoluto (Euler Explícito) & Erro relativo (Euler Explícito) & Erro absoluto (Euler Implícito) & Erro relativo (Euler Implícito) & Erro absoluto (Método dos Trapézios) & Erro relativo (Método dos Trapézios)\\
\midrule
0.00e+00 & 0.00e+00 & 0.00e+00 & 0.00e+00 & 0.00e+00 & 0.00e+00 & 0.00e+00\\
1.00e-01 & 9.05e-05 & 9.95e-05 & 9.04e-05 & 9.94e-05 & 3.02e-08 & 3.32e-08\\
2.00e-01 & 1.64e-04 & 1.96e-04 & 1.64e-04 & 1.95e-04 & 5.46e-08 & 6.52e-08\\
3.00e-01 & 2.22e-04 & 2.85e-04 & 2.22e-04 & 2.84e-04 & 7.41e-08 & 9.48e-08\\
4.00e-01 & 2.68e-04 & 3.62e-04 & 2.68e-04 & 3.62e-04 & 8.94e-08 & 1.21e-07\\
5.00e-01 & 3.03e-04 & 4.26e-04 & 3.03e-04 & 4.25e-04 & 1.01e-07 & 1.42e-07\\
6.00e-01 & 3.29e-04 & 4.72e-04 & 3.29e-04 & 4.72e-04 & 1.10e-07 & 1.57e-07\\
7.00e-01 & 3.48e-04 & 5.02e-04 & 3.47e-04 & 5.01e-04 & 1.16e-07 & 1.67e-07\\
8.00e-01 & 3.60e-04 & 5.15e-04 & 3.59e-04 & 5.14e-04 & 1.20e-07 & 1.72e-07\\
9.00e-01 & 3.66e-04 & 5.13e-04 & 3.66e-04 & 5.13e-04 & 1.22e-07 & 1.71e-07\\
1.00e+00 & 3.68e-04 & 5.00e-04 & 3.68e-04 & 5.00e-04 & 1.23e-07 & 1.67e-07\\
\bottomrule
\end{tabular}
}
    \caption{Erros obtidos para o item $a$ com $h=0.001$}
\end{table}

\begin{figure}[H]
    \includegraphics[width=\linewidth]{results/ex1/a/h_0.001.png}
    \caption{Gráfico plotado para o item $a$ quando $h=0.001$}
\end{figure}

\begin{figure}[H]
    \includegraphics[width=\linewidth]{results/ex1/a/h_0.001_abs_error.png}
    \caption{Gráfico plotado para o erro absoluto do item $a$ quando $h=0.001$}
\end{figure}

\begin{figure}[H]
    \includegraphics[width=\linewidth]{results/ex1/a/h_0.001_rel_error.png}
    \caption{Gráfico plotado para o erro relativo do item $a$ quando $h=0.001$}
\end{figure}
\subsection{Exercício 1, item $b$}\subsubsection{$h=0.1$}

\begin{table}[H]
    \centering
    \resizebox{\textwidth}{!}{
\begin{tabular}{ccccc}
\toprule
x & Valor real & Valor (Euler Explícito) & Valor (Euler Implícito) & Valor (Método dos Trapézios)\\
\midrule
-2.5000 & 0.0019 & 0.0019 & 0.0019 & 0.0019\\
-2.0000 & 0.0183 & 0.0128 & 0.0353 & 0.0173\\
-1.5000 & 0.1054 & 0.0595 & 0.2830 & 0.0970\\
-1.0000 & 0.3679 & 0.1886 & 1.1200 & 0.3346\\
-0.5000 & 0.7788 & 0.3955 & 2.3872 & 0.7062\\
0.0000 & 1.0000 & 0.5284 & 2.9341 & 0.9064\\
0.5000 & 0.7788 & 0.4299 & 2.1965 & 0.7059\\
1.0000 & 0.3679 & 0.2017 & 1.0473 & 0.3345\\
1.5000 & 0.1054 & 0.0510 & 0.3302 & 0.0974\\
2.0000 & 0.0183 & 0.0064 & 0.0710 & 0.0177\\
2.5000 & 0.0019 & 0.0003 & 0.0107 & 0.0021\\
\bottomrule
\end{tabular}
}
    \caption{Valores obtidos para o item $b$ com $h=0.1$}
\end{table}

\begin{table}[H]
    \centering
    \resizebox{\textwidth}{!}{
\begin{tabular}{ccccccc}
\toprule
x & Erro absoluto (Euler Explícito) & Erro relativo (Euler Explícito) & Erro absoluto (Euler Implícito) & Erro relativo (Euler Implícito) & Erro absoluto (Método dos Trapézios) & Erro relativo (Método dos Trapézios)\\
\midrule
-2.50e+00 & 0.00e+00 & 0.00e+00 & 0.00e+00 & 0.00e+00 & 0.00e+00 & 0.00e+00\\
-2.00e+00 & 5.52e-03 & 3.01e-01 & 1.70e-02 & 9.26e-01 & 9.86e-04 & 5.38e-02\\
-1.50e+00 & 4.59e-02 & 4.36e-01 & 1.78e-01 & 1.68e+00 & 8.45e-03 & 8.01e-02\\
-1.00e+00 & 1.79e-01 & 4.87e-01 & 7.52e-01 & 2.04e+00 & 3.33e-02 & 9.05e-02\\
-5.00e-01 & 3.83e-01 & 4.92e-01 & 1.61e+00 & 2.07e+00 & 7.26e-02 & 9.32e-02\\
0.00e+00 & 4.72e-01 & 4.72e-01 & 1.93e+00 & 1.93e+00 & 9.36e-02 & 9.36e-02\\
5.00e-01 & 3.49e-01 & 4.48e-01 & 1.42e+00 & 1.82e+00 & 7.29e-02 & 9.36e-02\\
1.00e+00 & 1.66e-01 & 4.52e-01 & 6.79e-01 & 1.85e+00 & 3.34e-02 & 9.07e-02\\
1.50e+00 & 5.44e-02 & 5.16e-01 & 2.25e-01 & 2.13e+00 & 8.01e-03 & 7.60e-02\\
2.00e+00 & 1.20e-02 & 6.53e-01 & 5.27e-02 & 2.88e+00 & 5.74e-04 & 3.14e-02\\
2.50e+00 & 1.58e-03 & 8.20e-01 & 8.79e-03 & 4.55e+00 & 1.51e-04 & 7.80e-02\\
\bottomrule
\end{tabular}
}
    \caption{Erros obtidos para o item $b$ com $h=0.1$}
\end{table}

\begin{figure}[H]
    \includegraphics[width=\linewidth]{results/ex1/b/h_0.1.png}
    \caption{Gráfico plotado para o item $b$ quando $h=0.1$}
\end{figure}

\begin{figure}[H]
    \includegraphics[width=\linewidth]{results/ex1/b/h_0.1_abs_error.png}
    \caption{Gráfico plotado para o erro absoluto do item $b$ quando $h=0.1$}
\end{figure}

\begin{figure}[H]
    \includegraphics[width=\linewidth]{results/ex1/b/h_0.1_rel_error.png}
    \caption{Gráfico plotado para o erro relativo do item $b$ quando $h=0.1$}
\end{figure}
\subsubsection{$h=0.01$}

\begin{table}[H]
    \centering
    \resizebox{\textwidth}{!}{
\begin{tabular}{ccccc}
\toprule
x & Valor real & Valor (Euler Explícito) & Valor (Euler Implícito) & Valor (Método dos Trapézios)\\
\midrule
-2.5000 & 0.0019 & 0.0019 & 0.0019 & 0.0019\\
-2.0000 & 0.0183 & 0.0175 & 0.0192 & 0.0183\\
-1.5000 & 0.1054 & 0.0983 & 0.1134 & 0.1053\\
-1.0000 & 0.3679 & 0.3394 & 0.4003 & 0.3674\\
-0.5000 & 0.7788 & 0.7179 & 0.8482 & 0.7778\\
0.0000 & 1.0000 & 0.9257 & 1.0846 & 0.9987\\
0.5000 & 0.7788 & 0.7239 & 0.8412 & 0.7778\\
1.0000 & 0.3679 & 0.3417 & 0.3977 & 0.3674\\
1.5000 & 0.1054 & 0.0968 & 0.1152 & 0.1053\\
2.0000 & 0.0183 & 0.0164 & 0.0205 & 0.0183\\
2.5000 & 0.0019 & 0.0016 & 0.0023 & 0.0019\\
\bottomrule
\end{tabular}
}
    \caption{Valores obtidos para o item $b$ com $h=0.01$}
\end{table}

\begin{table}[H]
    \centering
    \resizebox{\textwidth}{!}{
\begin{tabular}{ccccccc}
\toprule
x & Erro absoluto (Euler Explícito) & Erro relativo (Euler Explícito) & Erro absoluto (Euler Implícito) & Erro relativo (Euler Implícito) & Erro absoluto (Método dos Trapézios) & Erro relativo (Método dos Trapézios)\\
\midrule
-2.50e+00 & 0.00e+00 & 0.00e+00 & 0.00e+00 & 0.00e+00 & 0.00e+00 & 0.00e+00\\
-2.00e+00 & 7.98e-04 & 4.36e-02 & 8.85e-04 & 4.83e-02 & 1.36e-05 & 7.43e-04\\
-1.50e+00 & 7.11e-03 & 6.75e-02 & 8.05e-03 & 7.64e-02 & 1.16e-04 & 1.10e-03\\
-1.00e+00 & 2.85e-02 & 7.74e-02 & 3.25e-02 & 8.83e-02 & 4.53e-04 & 1.23e-03\\
-5.00e-01 & 6.09e-02 & 7.82e-02 & 6.94e-02 & 8.91e-02 & 9.83e-04 & 1.26e-03\\
0.00e+00 & 7.43e-02 & 7.43e-02 & 8.46e-02 & 8.46e-02 & 1.26e-03 & 1.26e-03\\
5.00e-01 & 5.49e-02 & 7.05e-02 & 6.24e-02 & 8.01e-02 & 9.83e-04 & 1.26e-03\\
1.00e+00 & 2.62e-02 & 7.12e-02 & 2.98e-02 & 8.10e-02 & 4.53e-04 & 1.23e-03\\
1.50e+00 & 8.58e-03 & 8.14e-02 & 9.76e-03 & 9.26e-02 & 1.15e-04 & 1.09e-03\\
2.00e+00 & 1.93e-03 & 1.05e-01 & 2.21e-03 & 1.21e-01 & 1.32e-05 & 7.19e-04\\
2.50e+00 & 2.83e-04 & 1.47e-01 & 3.31e-04 & 1.72e-01 & 1.46e-07 & 7.56e-05\\
\bottomrule
\end{tabular}
}
    \caption{Erros obtidos para o item $b$ com $h=0.01$}
\end{table}

\begin{figure}[H]
    \includegraphics[width=\linewidth]{results/ex1/b/h_0.01.png}
    \caption{Gráfico plotado para o item $b$ quando $h=0.01$}
\end{figure}

\begin{figure}[H]
    \includegraphics[width=\linewidth]{results/ex1/b/h_0.01_abs_error.png}
    \caption{Gráfico plotado para o erro absoluto do item $b$ quando $h=0.01$}
\end{figure}

\begin{figure}[H]
    \includegraphics[width=\linewidth]{results/ex1/b/h_0.01_rel_error.png}
    \caption{Gráfico plotado para o erro relativo do item $b$ quando $h=0.01$}
\end{figure}
\subsubsection{$h=0.005$}

\begin{table}[H]
    \centering
    \resizebox{\textwidth}{!}{
\begin{tabular}{ccccc}
\toprule
x & Valor real & Valor (Euler Explícito) & Valor (Euler Implícito) & Valor (Método dos Trapézios)\\
\midrule
-2.5000 & 0.0019 & 0.0019 & 0.0019 & 0.0019\\
-2.0000 & 0.0183 & 0.0179 & 0.0187 & 0.0183\\
-1.5000 & 0.1054 & 0.1017 & 0.1093 & 0.1054\\
-1.0000 & 0.3679 & 0.3532 & 0.3836 & 0.3678\\
-0.5000 & 0.7788 & 0.7474 & 0.8123 & 0.7786\\
0.0000 & 1.0000 & 0.9617 & 1.0409 & 0.9997\\
0.5000 & 0.7788 & 0.7505 & 0.8090 & 0.7786\\
1.0000 & 0.3679 & 0.3544 & 0.3823 & 0.3678\\
1.5000 & 0.1054 & 0.1010 & 0.1101 & 0.1054\\
2.0000 & 0.0183 & 0.0173 & 0.0194 & 0.0183\\
2.5000 & 0.0019 & 0.0018 & 0.0021 & 0.0019\\
\bottomrule
\end{tabular}
}
    \caption{Valores obtidos para o item $b$ com $h=0.005$}
\end{table}

\begin{table}[H]
    \centering
    \resizebox{\textwidth}{!}{
\begin{tabular}{ccccccc}
\toprule
x & Erro absoluto (Euler Explícito) & Erro relativo (Euler Explícito) & Erro absoluto (Euler Implícito) & Erro relativo (Euler Implícito) & Erro absoluto (Método dos Trapézios) & Erro relativo (Método dos Trapézios)\\
\midrule
-2.50e+00 & 0.00e+00 & 0.00e+00 & 0.00e+00 & 0.00e+00 & 0.00e+00 & 0.00e+00\\
-2.00e+00 & 4.09e-04 & 2.23e-02 & 4.31e-04 & 2.35e-02 & 3.46e-06 & 1.89e-04\\
-1.50e+00 & 3.66e-03 & 3.48e-02 & 3.90e-03 & 3.70e-02 & 2.94e-05 & 2.79e-04\\
-1.00e+00 & 1.47e-02 & 3.99e-02 & 1.57e-02 & 4.26e-02 & 1.15e-04 & 3.12e-04\\
-5.00e-01 & 3.14e-02 & 4.03e-02 & 3.35e-02 & 4.31e-02 & 2.49e-04 & 3.20e-04\\
0.00e+00 & 3.83e-02 & 3.83e-02 & 4.09e-02 & 4.09e-02 & 3.21e-04 & 3.21e-04\\
5.00e-01 & 2.83e-02 & 3.63e-02 & 3.02e-02 & 3.87e-02 & 2.49e-04 & 3.20e-04\\
1.00e+00 & 1.35e-02 & 3.67e-02 & 1.44e-02 & 3.92e-02 & 1.15e-04 & 3.12e-04\\
1.50e+00 & 4.42e-03 & 4.20e-02 & 4.72e-03 & 4.48e-02 & 2.93e-05 & 2.78e-04\\
2.00e+00 & 9.96e-04 & 5.44e-02 & 1.07e-03 & 5.82e-02 & 3.41e-06 & 1.86e-04\\
2.50e+00 & 1.47e-04 & 7.61e-02 & 1.59e-04 & 8.24e-02 & 1.82e-08 & 9.45e-06\\
\bottomrule
\end{tabular}
}
    \caption{Erros obtidos para o item $b$ com $h=0.005$}
\end{table}

\begin{figure}[H]
    \includegraphics[width=\linewidth]{results/ex1/b/h_0.005.png}
    \caption{Gráfico plotado para o item $b$ quando $h=0.005$}
\end{figure}

\begin{figure}[H]
    \includegraphics[width=\linewidth]{results/ex1/b/h_0.005_abs_error.png}
    \caption{Gráfico plotado para o erro absoluto do item $b$ quando $h=0.005$}
\end{figure}

\begin{figure}[H]
    \includegraphics[width=\linewidth]{results/ex1/b/h_0.005_rel_error.png}
    \caption{Gráfico plotado para o erro relativo do item $b$ quando $h=0.005$}
\end{figure}
\subsubsection{$h=0.001$}

\begin{table}[H]
    \centering
    \resizebox{\textwidth}{!}{
\begin{tabular}{ccccc}
\toprule
x & Valor real & Valor (Euler Explícito) & Valor (Euler Implícito) & Valor (Método dos Trapézios)\\
\midrule
-2.5000 & 0.0019 & 0.0019 & 0.0019 & 0.0019\\
-2.0000 & 0.0183 & 0.0182 & 0.0184 & 0.0183\\
-1.5000 & 0.1054 & 0.1046 & 0.1062 & 0.1054\\
-1.0000 & 0.3679 & 0.3649 & 0.3709 & 0.3679\\
-0.5000 & 0.7788 & 0.7724 & 0.7853 & 0.7788\\
0.0000 & 1.0000 & 0.9921 & 1.0080 & 1.0000\\
0.5000 & 0.7788 & 0.7730 & 0.7847 & 0.7788\\
1.0000 & 0.3679 & 0.3651 & 0.3707 & 0.3679\\
1.5000 & 0.1054 & 0.1045 & 0.1063 & 0.1054\\
2.0000 & 0.0183 & 0.0181 & 0.0185 & 0.0183\\
2.5000 & 0.0019 & 0.0019 & 0.0020 & 0.0019\\
\bottomrule
\end{tabular}
}
    \caption{Valores obtidos para o item $b$ com $h=0.001$}
\end{table}

\begin{table}[H]
    \centering
    \resizebox{\textwidth}{!}{
\begin{tabular}{ccccccc}
\toprule
x & Erro absoluto (Euler Explícito) & Erro relativo (Euler Explícito) & Erro absoluto (Euler Implícito) & Erro relativo (Euler Implícito) & Erro absoluto (Método dos Trapézios) & Erro relativo (Método dos Trapézios)\\
\midrule
-2.50e+00 & 0.00e+00 & 0.00e+00 & 0.00e+00 & 0.00e+00 & 0.00e+00 & 0.00e+00\\
-2.00e+00 & 8.35e-05 & 4.56e-03 & 8.44e-05 & 4.61e-03 & 1.40e-07 & 7.66e-06\\
-1.50e+00 & 7.51e-04 & 7.12e-03 & 7.60e-04 & 7.21e-03 & 1.19e-06 & 1.13e-05\\
-1.00e+00 & 3.02e-03 & 8.20e-03 & 3.06e-03 & 8.30e-03 & 4.65e-06 & 1.26e-05\\
-5.00e-01 & 6.45e-03 & 8.28e-03 & 6.53e-03 & 8.39e-03 & 1.01e-05 & 1.30e-05\\
0.00e+00 & 7.87e-03 & 7.87e-03 & 7.97e-03 & 7.97e-03 & 1.30e-05 & 1.30e-05\\
5.00e-01 & 5.80e-03 & 7.45e-03 & 5.88e-03 & 7.55e-03 & 1.01e-05 & 1.30e-05\\
1.00e+00 & 2.77e-03 & 7.53e-03 & 2.81e-03 & 7.63e-03 & 4.65e-06 & 1.26e-05\\
1.50e+00 & 9.08e-04 & 8.61e-03 & 9.19e-04 & 8.72e-03 & 1.19e-06 & 1.13e-05\\
2.00e+00 & 2.05e-04 & 1.12e-02 & 2.07e-04 & 1.13e-02 & 1.40e-07 & 7.64e-06\\
2.50e+00 & 3.03e-05 & 1.57e-02 & 3.08e-05 & 1.60e-02 & 1.46e-10 & 7.56e-08\\
\bottomrule
\end{tabular}
}
    \caption{Erros obtidos para o item $b$ com $h=0.001$}
\end{table}

\begin{figure}[H]
    \includegraphics[width=\linewidth]{results/ex1/b/h_0.001.png}
    \caption{Gráfico plotado para o item $b$ quando $h=0.001$}
\end{figure}

\begin{figure}[H]
    \includegraphics[width=\linewidth]{results/ex1/b/h_0.001_abs_error.png}
    \caption{Gráfico plotado para o erro absoluto do item $b$ quando $h=0.001$}
\end{figure}

\begin{figure}[H]
    \includegraphics[width=\linewidth]{results/ex1/b/h_0.001_rel_error.png}
    \caption{Gráfico plotado para o erro relativo do item $b$ quando $h=0.001$}
\end{figure}
\subsection{Exercício 1, item $c$}\subsubsection{$h=0.1$}

\begin{table}[H]
    \centering
    \resizebox{\textwidth}{!}{
\begin{tabular}{ccccc}
\toprule
x & Valor real & Valor (Euler Explícito) & Valor (Euler Implícito) & Valor (Método dos Trapézios)\\
\midrule
0.0000 & 1.2000 & 1.2000 & 1.2000 & 1.2000\\
0.2000 & 2.9183 & 2.4500 & 4.2000 & 2.8406\\
0.4000 & 7.5891 & 5.2625 & 16.2000 & 7.1729\\
0.6000 & 20.2855 & 11.5906 & 64.2000 & 18.6128\\
0.8000 & 54.7982 & 25.8289 & 256.2000 & 48.8213\\
1.0000 & 148.6132 & 57.8650 & 1024.2000 & 128.5907\\
1.2000 & 403.6288 & 129.9463 & 4096.2000 & 339.2318\\
1.4000 & 1096.8332 & 292.1293 & 16384.2000 & 895.4557\\
1.6000 & 2981.1580 & 657.0408 & 65536.2000 & 2364.2347\\
1.8000 & 8103.2839 & 1478.0919 & 262144.2000 & 6242.7291\\
2.0000 & 22026.6658 & 3325.4567 & 1048576.2000 & 16484.3784\\
\bottomrule
\end{tabular}
}
    \caption{Valores obtidos para o item $c$ com $h=0.1$}
\end{table}

\begin{table}[H]
    \centering
    \resizebox{\textwidth}{!}{
\begin{tabular}{ccccccc}
\toprule
x & Erro absoluto (Euler Explícito) & Erro relativo (Euler Explícito) & Erro absoluto (Euler Implícito) & Erro relativo (Euler Implícito) & Erro absoluto (Método dos Trapézios) & Erro relativo (Método dos Trapézios)\\
\midrule
0.00e+00 & 0.00e+00 & 0.00e+00 & 0.00e+00 & 0.00e+00 & 0.00e+00 & 0.00e+00\\
2.00e-01 & 4.68e-01 & 1.60e-01 & 1.28e+00 & 4.39e-01 & 7.77e-02 & 2.66e-02\\
4.00e-01 & 2.33e+00 & 3.07e-01 & 8.61e+00 & 1.13e+00 & 4.16e-01 & 5.48e-02\\
6.00e-01 & 8.69e+00 & 4.29e-01 & 4.39e+01 & 2.16e+00 & 1.67e+00 & 8.25e-02\\
8.00e-01 & 2.90e+01 & 5.29e-01 & 2.01e+02 & 3.68e+00 & 5.98e+00 & 1.09e-01\\
1.00e+00 & 9.07e+01 & 6.11e-01 & 8.76e+02 & 5.89e+00 & 2.00e+01 & 1.35e-01\\
1.20e+00 & 2.74e+02 & 6.78e-01 & 3.69e+03 & 9.15e+00 & 6.44e+01 & 1.60e-01\\
1.40e+00 & 8.05e+02 & 7.34e-01 & 1.53e+04 & 1.39e+01 & 2.01e+02 & 1.84e-01\\
1.60e+00 & 2.32e+03 & 7.80e-01 & 6.26e+04 & 2.10e+01 & 6.17e+02 & 2.07e-01\\
1.80e+00 & 6.63e+03 & 8.18e-01 & 2.54e+05 & 3.14e+01 & 1.86e+03 & 2.30e-01\\
2.00e+00 & 1.87e+04 & 8.49e-01 & 1.03e+06 & 4.66e+01 & 5.54e+03 & 2.52e-01\\
\bottomrule
\end{tabular}
}
    \caption{Erros obtidos para o item $c$ com $h=0.1$}
\end{table}

\begin{figure}[H]
    \includegraphics[width=\linewidth]{results/ex1/c/h_0.1.png}
    \caption{Gráfico plotado para o item $c$ quando $h=0.1$}
\end{figure}

\begin{figure}[H]
    \includegraphics[width=\linewidth]{results/ex1/c/h_0.1_abs_error.png}
    \caption{Gráfico plotado para o erro absoluto do item $c$ quando $h=0.1$}
\end{figure}

\begin{figure}[H]
    \includegraphics[width=\linewidth]{results/ex1/c/h_0.1_rel_error.png}
    \caption{Gráfico plotado para o erro relativo do item $c$ quando $h=0.1$}
\end{figure}
\subsubsection{$h=0.01$}

\begin{table}[H]
    \centering
    \resizebox{\textwidth}{!}{
\begin{tabular}{ccccc}
\toprule
x & Valor real & Valor (Euler Explícito) & Valor (Euler Implícito) & Valor (Método dos Trapézios)\\
\midrule
0.0000 & 1.2000 & 1.2000 & 1.2000 & 1.2000\\
0.2000 & 2.9183 & 2.8533 & 2.9895 & 2.9172\\
0.4000 & 7.5891 & 7.2400 & 7.9814 & 7.5831\\
0.6000 & 20.2855 & 18.8792 & 21.9062 & 20.2614\\
0.8000 & 54.7982 & 49.7614 & 60.7496 & 54.7106\\
1.0000 & 148.6132 & 131.7013 & 169.1038 & 148.3156\\
1.2000 & 403.6288 & 349.1120 & 471.3589 & 402.6585\\
1.4000 & 1096.8332 & 925.9674 & 1314.5023 & 1093.7565\\
1.6000 & 2981.1580 & 2456.5364 & 3666.4591 & 2971.6020\\
1.8000 & 8103.2839 & 6517.5918 & 10227.2657 & 8074.0669\\
2.0000 & 22026.6658 & 17292.7808 & 28528.7003 & 21938.4389\\
\bottomrule
\end{tabular}
}
    \caption{Valores obtidos para o item $c$ com $h=0.01$}
\end{table}

\begin{table}[H]
    \centering
    \resizebox{\textwidth}{!}{
\begin{tabular}{ccccccc}
\toprule
x & Erro absoluto (Euler Explícito) & Erro relativo (Euler Explícito) & Erro absoluto (Euler Implícito) & Erro relativo (Euler Implícito) & Erro absoluto (Método dos Trapézios) & Erro relativo (Método dos Trapézios)\\
\midrule
0.00e+00 & 0.00e+00 & 0.00e+00 & 0.00e+00 & 0.00e+00 & 0.00e+00 & 0.00e+00\\
2.00e-01 & 6.50e-02 & 2.23e-02 & 7.12e-02 & 2.44e-02 & 1.09e-03 & 3.74e-04\\
4.00e-01 & 3.49e-01 & 4.60e-02 & 3.92e-01 & 5.17e-02 & 5.93e-03 & 7.81e-04\\
6.00e-01 & 1.41e+00 & 6.93e-02 & 1.62e+00 & 7.99e-02 & 2.42e-02 & 1.19e-03\\
8.00e-01 & 5.04e+00 & 9.19e-02 & 5.95e+00 & 1.09e-01 & 8.76e-02 & 1.60e-03\\
1.00e+00 & 1.69e+01 & 1.14e-01 & 2.05e+01 & 1.38e-01 & 2.98e-01 & 2.00e-03\\
1.20e+00 & 5.45e+01 & 1.35e-01 & 6.77e+01 & 1.68e-01 & 9.70e-01 & 2.40e-03\\
1.40e+00 & 1.71e+02 & 1.56e-01 & 2.18e+02 & 1.98e-01 & 3.08e+00 & 2.81e-03\\
1.60e+00 & 5.25e+02 & 1.76e-01 & 6.85e+02 & 2.30e-01 & 9.56e+00 & 3.21e-03\\
1.80e+00 & 1.59e+03 & 1.96e-01 & 2.12e+03 & 2.62e-01 & 2.92e+01 & 3.61e-03\\
2.00e+00 & 4.73e+03 & 2.15e-01 & 6.50e+03 & 2.95e-01 & 8.82e+01 & 4.01e-03\\
\bottomrule
\end{tabular}
}
    \caption{Erros obtidos para o item $c$ com $h=0.01$}
\end{table}

\begin{figure}[H]
    \includegraphics[width=\linewidth]{results/ex1/c/h_0.01.png}
    \caption{Gráfico plotado para o item $c$ quando $h=0.01$}
\end{figure}

\begin{figure}[H]
    \includegraphics[width=\linewidth]{results/ex1/c/h_0.01_abs_error.png}
    \caption{Gráfico plotado para o erro absoluto do item $c$ quando $h=0.01$}
\end{figure}

\begin{figure}[H]
    \includegraphics[width=\linewidth]{results/ex1/c/h_0.01_rel_error.png}
    \caption{Gráfico plotado para o erro relativo do item $c$ quando $h=0.01$}
\end{figure}
\subsubsection{$h=0.005$}

\begin{table}[H]
    \centering
    \resizebox{\textwidth}{!}{
\begin{tabular}{ccccc}
\toprule
x & Valor real & Valor (Euler Explícito) & Valor (Euler Implícito) & Valor (Método dos Trapézios)\\
\midrule
0.0000 & 1.2000 & 1.2000 & 1.2000 & 1.2000\\
0.2000 & 2.9183 & 2.8851 & 2.9531 & 2.9180\\
0.4000 & 7.5891 & 7.4096 & 7.7793 & 7.5875\\
0.6000 & 20.2855 & 19.5581 & 21.0663 & 20.2794\\
0.8000 & 54.7982 & 52.1779 & 57.6462 & 54.7758\\
1.0000 & 148.6132 & 139.7639 & 158.3528 & 148.5373\\
1.2000 & 403.6288 & 374.9380 & 435.6038 & 403.3814\\
1.4000 & 1096.8332 & 1006.3954 & 1198.8920 & 1096.0487\\
1.6000 & 2981.1580 & 2701.8988 & 3300.2687 & 2978.7210\\
1.8000 & 8103.2839 & 7254.4337 & 9085.4807 & 8095.8317\\
2.0000 & 22026.6658 & 19478.2805 & 25012.5053 & 22004.1590\\
\bottomrule
\end{tabular}
}
    \caption{Valores obtidos para o item $c$ com $h=0.005$}
\end{table}

\begin{table}[H]
    \centering
    \resizebox{\textwidth}{!}{
\begin{tabular}{ccccccc}
\toprule
x & Erro absoluto (Euler Explícito) & Erro relativo (Euler Explícito) & Erro absoluto (Euler Implícito) & Erro relativo (Euler Implícito) & Erro absoluto (Método dos Trapézios) & Erro relativo (Método dos Trapézios)\\
\midrule
0.00e+00 & 0.00e+00 & 0.00e+00 & 0.00e+00 & 0.00e+00 & 0.00e+00 & 0.00e+00\\
2.00e-01 & 3.32e-02 & 1.14e-02 & 3.48e-02 & 1.19e-02 & 2.78e-04 & 9.52e-05\\
4.00e-01 & 1.79e-01 & 2.37e-02 & 1.90e-01 & 2.51e-02 & 1.51e-03 & 1.99e-04\\
6.00e-01 & 7.27e-01 & 3.59e-02 & 7.81e-01 & 3.85e-02 & 6.16e-03 & 3.04e-04\\
8.00e-01 & 2.62e+00 & 4.78e-02 & 2.85e+00 & 5.20e-02 & 2.23e-02 & 4.07e-04\\
1.00e+00 & 8.85e+00 & 5.95e-02 & 9.74e+00 & 6.55e-02 & 7.58e-02 & 5.10e-04\\
1.20e+00 & 2.87e+01 & 7.11e-02 & 3.20e+01 & 7.92e-02 & 2.47e-01 & 6.13e-04\\
1.40e+00 & 9.04e+01 & 8.25e-02 & 1.02e+02 & 9.30e-02 & 7.85e-01 & 7.15e-04\\
1.60e+00 & 2.79e+02 & 9.37e-02 & 3.19e+02 & 1.07e-01 & 2.44e+00 & 8.17e-04\\
1.80e+00 & 8.49e+02 & 1.05e-01 & 9.82e+02 & 1.21e-01 & 7.45e+00 & 9.20e-04\\
2.00e+00 & 2.55e+03 & 1.16e-01 & 2.99e+03 & 1.36e-01 & 2.25e+01 & 1.02e-03\\
\bottomrule
\end{tabular}
}
    \caption{Erros obtidos para o item $c$ com $h=0.005$}
\end{table}

\begin{figure}[H]
    \includegraphics[width=\linewidth]{results/ex1/c/h_0.005.png}
    \caption{Gráfico plotado para o item $c$ quando $h=0.005$}
\end{figure}

\begin{figure}[H]
    \includegraphics[width=\linewidth]{results/ex1/c/h_0.005_abs_error.png}
    \caption{Gráfico plotado para o erro absoluto do item $c$ quando $h=0.005$}
\end{figure}

\begin{figure}[H]
    \includegraphics[width=\linewidth]{results/ex1/c/h_0.005_rel_error.png}
    \caption{Gráfico plotado para o erro relativo do item $c$ quando $h=0.005$}
\end{figure}
\subsubsection{$h=0.001$}

\begin{table}[H]
    \centering
    \resizebox{\textwidth}{!}{
\begin{tabular}{ccccc}
\toprule
x & Valor real & Valor (Euler Explícito) & Valor (Euler Implícito) & Valor (Método dos Trapézios)\\
\midrule
0.0000 & 1.2000 & 1.2000 & 1.2000 & 1.2000\\
0.2000 & 2.9183 & 2.9115 & 2.9251 & 2.9183\\
0.4000 & 7.5891 & 7.5523 & 7.6262 & 7.5890\\
0.6000 & 20.2855 & 20.1360 & 20.4373 & 20.2853\\
0.8000 & 54.7982 & 54.2567 & 55.3487 & 54.7972\\
1.0000 & 148.6132 & 146.7756 & 150.4863 & 148.6101\\
1.2000 & 403.6288 & 397.6423 & 409.7464 & 403.6187\\
1.4000 & 1096.8332 & 1077.8717 & 1116.2585 & 1096.8013\\
1.6000 & 2981.1580 & 2922.3251 & 3041.5808 & 2981.0590\\
1.8000 & 8103.2839 & 7923.5923 & 8288.2938 & 8102.9812\\
2.0000 & 22026.6658 & 21484.6140 & 22586.1576 & 22025.7515\\
\bottomrule
\end{tabular}
}
    \caption{Valores obtidos para o item $c$ com $h=0.001$}
\end{table}

\begin{table}[H]
    \centering
    \resizebox{\textwidth}{!}{
\begin{tabular}{ccccccc}
\toprule
x & Erro absoluto (Euler Explícito) & Erro relativo (Euler Explícito) & Erro absoluto (Euler Implícito) & Erro relativo (Euler Implícito) & Erro absoluto (Método dos Trapézios) & Erro relativo (Método dos Trapézios)\\
\midrule
0.00e+00 & 0.00e+00 & 0.00e+00 & 0.00e+00 & 0.00e+00 & 0.00e+00 & 0.00e+00\\
2.00e-01 & 6.76e-03 & 2.32e-03 & 6.83e-03 & 2.34e-03 & 1.13e-05 & 3.87e-06\\
4.00e-01 & 3.67e-02 & 4.84e-03 & 3.72e-02 & 4.90e-03 & 6.13e-05 & 8.08e-06\\
6.00e-01 & 1.50e-01 & 7.37e-03 & 1.52e-01 & 7.48e-03 & 2.50e-04 & 1.23e-05\\
8.00e-01 & 5.41e-01 & 9.88e-03 & 5.51e-01 & 1.00e-02 & 9.07e-04 & 1.65e-05\\
1.00e+00 & 1.84e+00 & 1.24e-02 & 1.87e+00 & 1.26e-02 & 3.08e-03 & 2.07e-05\\
1.20e+00 & 5.99e+00 & 1.48e-02 & 6.12e+00 & 1.52e-02 & 1.00e-02 & 2.49e-05\\
1.40e+00 & 1.90e+01 & 1.73e-02 & 1.94e+01 & 1.77e-02 & 3.19e-02 & 2.91e-05\\
1.60e+00 & 5.88e+01 & 1.97e-02 & 6.04e+01 & 2.03e-02 & 9.90e-02 & 3.32e-05\\
1.80e+00 & 1.80e+02 & 2.22e-02 & 1.85e+02 & 2.28e-02 & 3.03e-01 & 3.74e-05\\
2.00e+00 & 5.42e+02 & 2.46e-02 & 5.59e+02 & 2.54e-02 & 9.14e-01 & 4.15e-05\\
\bottomrule
\end{tabular}
}
    \caption{Erros obtidos para o item $c$ com $h=0.001$}
\end{table}

\begin{figure}[H]
    \includegraphics[width=\linewidth]{results/ex1/c/h_0.001.png}
    \caption{Gráfico plotado para o item $c$ quando $h=0.001$}
\end{figure}

\begin{figure}[H]
    \includegraphics[width=\linewidth]{results/ex1/c/h_0.001_abs_error.png}
    \caption{Gráfico plotado para o erro absoluto do item $c$ quando $h=0.001$}
\end{figure}

\begin{figure}[H]
    \includegraphics[width=\linewidth]{results/ex1/c/h_0.001_rel_error.png}
    \caption{Gráfico plotado para o erro relativo do item $c$ quando $h=0.001$}
\end{figure}
\subsection{Exercício 1, item $d$}\subsubsection{$h=0.1$}

\begin{table}[H]
    \centering
    \resizebox{\textwidth}{!}{
\begin{tabular}{ccccc}
\toprule
x & Valor real & Valor (Euler Explícito) & Valor (Euler Implícito) & Valor (Método dos Trapézios)\\
\midrule
0.0000 & 0.5000 & 0.5000 & 0.5000 & 0.5000\\
0.2000 & 0.8293 & 0.8140 & 0.8462 & 0.8284\\
0.4000 & 1.2141 & 1.1815 & 1.2503 & 1.2122\\
0.6000 & 1.6489 & 1.5971 & 1.7073 & 1.6459\\
0.8000 & 2.1272 & 2.0538 & 2.2108 & 2.1228\\
1.0000 & 2.6409 & 2.5438 & 2.7528 & 2.6348\\
1.2000 & 3.1799 & 3.0569 & 3.3237 & 3.1720\\
1.4000 & 3.7324 & 3.5815 & 3.9115 & 3.7223\\
1.6000 & 4.2835 & 4.1030 & 4.5014 & 4.2708\\
1.8000 & 4.8152 & 4.6040 & 5.0750 & 4.7996\\
2.0000 & 5.3055 & 5.0635 & 5.6099 & 5.2866\\
\bottomrule
\end{tabular}
}
    \caption{Valores obtidos para o item $d$ com $h=0.1$}
\end{table}

\begin{table}[H]
    \centering
    \resizebox{\textwidth}{!}{
\begin{tabular}{ccccccc}
\toprule
x & Erro absoluto (Euler Explícito) & Erro relativo (Euler Explícito) & Erro absoluto (Euler Implícito) & Erro relativo (Euler Implícito) & Erro absoluto (Método dos Trapézios) & Erro relativo (Método dos Trapézios)\\
\midrule
0.00e+00 & 0.00e+00 & 0.00e+00 & 0.00e+00 & 0.00e+00 & 0.00e+00 & 0.00e+00\\
2.00e-01 & 1.53e-02 & 1.84e-02 & 1.69e-02 & 2.03e-02 & 8.64e-04 & 1.04e-03\\
4.00e-01 & 3.25e-02 & 2.68e-02 & 3.62e-02 & 2.99e-02 & 1.88e-03 & 1.55e-03\\
6.00e-01 & 5.19e-02 & 3.15e-02 & 5.84e-02 & 3.54e-02 & 3.06e-03 & 1.86e-03\\
8.00e-01 & 7.34e-02 & 3.45e-02 & 8.35e-02 & 3.93e-02 & 4.45e-03 & 2.09e-03\\
1.00e+00 & 9.71e-02 & 3.68e-02 & 1.12e-01 & 4.24e-02 & 6.06e-03 & 2.30e-03\\
1.20e+00 & 1.23e-01 & 3.87e-02 & 1.44e-01 & 4.52e-02 & 7.94e-03 & 2.50e-03\\
1.40e+00 & 1.51e-01 & 4.04e-02 & 1.79e-01 & 4.80e-02 & 1.01e-02 & 2.71e-03\\
1.60e+00 & 1.80e-01 & 4.21e-02 & 2.18e-01 & 5.09e-02 & 1.26e-02 & 2.95e-03\\
1.80e+00 & 2.11e-01 & 4.38e-02 & 2.60e-01 & 5.40e-02 & 1.56e-02 & 3.23e-03\\
2.00e+00 & 2.42e-01 & 4.56e-02 & 3.04e-01 & 5.74e-02 & 1.89e-02 & 3.56e-03\\
\bottomrule
\end{tabular}
}
    \caption{Erros obtidos para o item $d$ com $h=0.1$}
\end{table}

\begin{figure}[H]
    \includegraphics[width=\linewidth]{results/ex1/d/h_0.1.png}
    \caption{Gráfico plotado para o item $d$ quando $h=0.1$}
\end{figure}

\begin{figure}[H]
    \includegraphics[width=\linewidth]{results/ex1/d/h_0.1_abs_error.png}
    \caption{Gráfico plotado para o erro absoluto do item $d$ quando $h=0.1$}
\end{figure}

\begin{figure}[H]
    \includegraphics[width=\linewidth]{results/ex1/d/h_0.1_rel_error.png}
    \caption{Gráfico plotado para o erro relativo do item $d$ quando $h=0.1$}
\end{figure}
\subsubsection{$h=0.01$}

\begin{table}[H]
    \centering
    \resizebox{\textwidth}{!}{
\begin{tabular}{ccccc}
\toprule
x & Valor real & Valor (Euler Explícito) & Valor (Euler Implícito) & Valor (Método dos Trapézios)\\
\midrule
0.0000 & 0.5000 & 0.5000 & 0.5000 & 0.5000\\
0.2000 & 0.8293 & 0.8277 & 0.8309 & 0.8293\\
0.4000 & 1.2141 & 1.2107 & 1.2175 & 1.2141\\
0.6000 & 1.6489 & 1.6435 & 1.6545 & 1.6489\\
0.8000 & 2.1272 & 2.1195 & 2.1351 & 2.1272\\
1.0000 & 2.6409 & 2.6305 & 2.6513 & 2.6408\\
1.2000 & 3.1799 & 3.1668 & 3.1933 & 3.1799\\
1.4000 & 3.7324 & 3.7162 & 3.7489 & 3.7323\\
1.6000 & 4.2835 & 4.2639 & 4.3034 & 4.2834\\
1.8000 & 4.8152 & 4.7921 & 4.8387 & 4.8150\\
2.0000 & 5.3055 & 5.2788 & 5.3327 & 5.3053\\
\bottomrule
\end{tabular}
}
    \caption{Valores obtidos para o item $d$ com $h=0.01$}
\end{table}

\begin{table}[H]
    \centering
    \resizebox{\textwidth}{!}{
\begin{tabular}{ccccccc}
\toprule
x & Erro absoluto (Euler Explícito) & Erro relativo (Euler Explícito) & Erro absoluto (Euler Implícito) & Erro relativo (Euler Implícito) & Erro absoluto (Método dos Trapézios) & Erro relativo (Método dos Trapézios)\\
\midrule
0.00e+00 & 0.00e+00 & 0.00e+00 & 0.00e+00 & 0.00e+00 & 0.00e+00 & 0.00e+00\\
2.00e-01 & 1.60e-03 & 1.92e-03 & 1.61e-03 & 1.94e-03 & 8.99e-06 & 1.08e-05\\
4.00e-01 & 3.41e-03 & 2.81e-03 & 3.44e-03 & 2.84e-03 & 1.95e-05 & 1.61e-05\\
6.00e-01 & 5.46e-03 & 3.31e-03 & 5.52e-03 & 3.35e-03 & 3.19e-05 & 1.93e-05\\
8.00e-01 & 7.75e-03 & 3.65e-03 & 7.86e-03 & 3.69e-03 & 4.62e-05 & 2.17e-05\\
1.00e+00 & 1.03e-02 & 3.91e-03 & 1.05e-02 & 3.96e-03 & 6.30e-05 & 2.39e-05\\
1.20e+00 & 1.31e-02 & 4.13e-03 & 1.33e-02 & 4.20e-03 & 8.25e-05 & 2.59e-05\\
1.40e+00 & 1.62e-02 & 4.35e-03 & 1.65e-02 & 4.42e-03 & 1.05e-04 & 2.81e-05\\
1.60e+00 & 1.95e-02 & 4.56e-03 & 1.99e-02 & 4.65e-03 & 1.31e-04 & 3.06e-05\\
1.80e+00 & 2.30e-02 & 4.78e-03 & 2.35e-02 & 4.88e-03 & 1.61e-04 & 3.35e-05\\
2.00e+00 & 2.66e-02 & 5.02e-03 & 2.73e-02 & 5.14e-03 & 1.96e-04 & 3.69e-05\\
\bottomrule
\end{tabular}
}
    \caption{Erros obtidos para o item $d$ com $h=0.01$}
\end{table}

\begin{figure}[H]
    \includegraphics[width=\linewidth]{results/ex1/d/h_0.01.png}
    \caption{Gráfico plotado para o item $d$ quando $h=0.01$}
\end{figure}

\begin{figure}[H]
    \includegraphics[width=\linewidth]{results/ex1/d/h_0.01_abs_error.png}
    \caption{Gráfico plotado para o erro absoluto do item $d$ quando $h=0.01$}
\end{figure}

\begin{figure}[H]
    \includegraphics[width=\linewidth]{results/ex1/d/h_0.01_rel_error.png}
    \caption{Gráfico plotado para o erro relativo do item $d$ quando $h=0.01$}
\end{figure}
\subsubsection{$h=0.005$}

\begin{table}[H]
    \centering
    \resizebox{\textwidth}{!}{
\begin{tabular}{ccccc}
\toprule
x & Valor real & Valor (Euler Explícito) & Valor (Euler Implícito) & Valor (Método dos Trapézios)\\
\midrule
0.0000 & 0.5000 & 0.5000 & 0.5000 & 0.5000\\
0.2000 & 0.8293 & 0.8285 & 0.8301 & 0.8293\\
0.4000 & 1.2141 & 1.2124 & 1.2158 & 1.2141\\
0.6000 & 1.6489 & 1.6462 & 1.6517 & 1.6489\\
0.8000 & 2.1272 & 2.1233 & 2.1311 & 2.1272\\
1.0000 & 2.6409 & 2.6357 & 2.6461 & 2.6408\\
1.2000 & 3.1799 & 3.1733 & 3.1866 & 3.1799\\
1.4000 & 3.7324 & 3.7243 & 3.7406 & 3.7324\\
1.6000 & 4.2835 & 4.2737 & 4.2934 & 4.2835\\
1.8000 & 4.8152 & 4.8036 & 4.8269 & 4.8151\\
2.0000 & 5.3055 & 5.2921 & 5.3190 & 5.3054\\
\bottomrule
\end{tabular}
}
    \caption{Valores obtidos para o item $d$ com $h=0.005$}
\end{table}

\begin{table}[H]
    \centering
    \resizebox{\textwidth}{!}{
\begin{tabular}{ccccccc}
\toprule
x & Erro absoluto (Euler Explícito) & Erro relativo (Euler Explícito) & Erro absoluto (Euler Implícito) & Erro relativo (Euler Implícito) & Erro absoluto (Método dos Trapézios) & Erro relativo (Método dos Trapézios)\\
\midrule
0.00e+00 & 0.00e+00 & 0.00e+00 & 0.00e+00 & 0.00e+00 & 0.00e+00 & 0.00e+00\\
2.00e-01 & 8.00e-04 & 9.64e-04 & 8.04e-04 & 9.69e-04 & 2.25e-06 & 2.72e-06\\
4.00e-01 & 1.71e-03 & 1.41e-03 & 1.72e-03 & 1.41e-03 & 4.89e-06 & 4.03e-06\\
6.00e-01 & 2.74e-03 & 1.66e-03 & 2.75e-03 & 1.67e-03 & 7.98e-06 & 4.84e-06\\
8.00e-01 & 3.89e-03 & 1.83e-03 & 3.91e-03 & 1.84e-03 & 1.16e-05 & 5.45e-06\\
1.00e+00 & 5.18e-03 & 1.96e-03 & 5.21e-03 & 1.97e-03 & 1.58e-05 & 5.98e-06\\
1.20e+00 & 6.59e-03 & 2.07e-03 & 6.65e-03 & 2.09e-03 & 2.07e-05 & 6.50e-06\\
1.40e+00 & 8.14e-03 & 2.18e-03 & 8.21e-03 & 2.20e-03 & 2.63e-05 & 7.05e-06\\
1.60e+00 & 9.81e-03 & 2.29e-03 & 9.91e-03 & 2.31e-03 & 3.28e-05 & 7.67e-06\\
1.80e+00 & 1.16e-02 & 2.40e-03 & 1.17e-02 & 2.43e-03 & 4.04e-05 & 8.38e-06\\
2.00e+00 & 1.34e-02 & 2.52e-03 & 1.36e-02 & 2.55e-03 & 4.90e-05 & 9.23e-06\\
\bottomrule
\end{tabular}
}
    \caption{Erros obtidos para o item $d$ com $h=0.005$}
\end{table}

\begin{figure}[H]
    \includegraphics[width=\linewidth]{results/ex1/d/h_0.005.png}
    \caption{Gráfico plotado para o item $d$ quando $h=0.005$}
\end{figure}

\begin{figure}[H]
    \includegraphics[width=\linewidth]{results/ex1/d/h_0.005_abs_error.png}
    \caption{Gráfico plotado para o erro absoluto do item $d$ quando $h=0.005$}
\end{figure}

\begin{figure}[H]
    \includegraphics[width=\linewidth]{results/ex1/d/h_0.005_rel_error.png}
    \caption{Gráfico plotado para o erro relativo do item $d$ quando $h=0.005$}
\end{figure}
\subsubsection{$h=0.001$}

\begin{table}[H]
    \centering
    \resizebox{\textwidth}{!}{
\begin{tabular}{ccccc}
\toprule
x & Valor real & Valor (Euler Explícito) & Valor (Euler Implícito) & Valor (Método dos Trapézios)\\
\midrule
0.0000 & 0.5000 & 0.5000 & 0.5000 & 0.5000\\
0.2000 & 0.8293 & 0.8291 & 0.8295 & 0.8293\\
0.4000 & 1.2141 & 1.2137 & 1.2144 & 1.2141\\
0.6000 & 1.6489 & 1.6484 & 1.6495 & 1.6489\\
0.8000 & 2.1272 & 2.1264 & 2.1280 & 2.1272\\
1.0000 & 2.6409 & 2.6398 & 2.6419 & 2.6409\\
1.2000 & 3.1799 & 3.1786 & 3.1813 & 3.1799\\
1.4000 & 3.7324 & 3.7308 & 3.7340 & 3.7324\\
1.6000 & 4.2835 & 4.2815 & 4.2855 & 4.2835\\
1.8000 & 4.8152 & 4.8129 & 4.8175 & 4.8152\\
2.0000 & 5.3055 & 5.3028 & 5.3082 & 5.3055\\
\bottomrule
\end{tabular}
}
    \caption{Valores obtidos para o item $d$ com $h=0.001$}
\end{table}

\begin{table}[H]
    \centering
    \resizebox{\textwidth}{!}{
\begin{tabular}{ccccccc}
\toprule
x & Erro absoluto (Euler Explícito) & Erro relativo (Euler Explícito) & Erro absoluto (Euler Implícito) & Erro relativo (Euler Implícito) & Erro absoluto (Método dos Trapézios) & Erro relativo (Método dos Trapézios)\\
\midrule
0.00e+00 & 0.00e+00 & 0.00e+00 & 0.00e+00 & 0.00e+00 & 0.00e+00 & 0.00e+00\\
2.00e-01 & 1.60e-04 & 1.93e-04 & 1.60e-04 & 1.93e-04 & 9.03e-08 & 1.09e-07\\
4.00e-01 & 3.42e-04 & 2.82e-04 & 3.43e-04 & 2.82e-04 & 1.96e-07 & 1.62e-07\\
6.00e-01 & 5.48e-04 & 3.33e-04 & 5.49e-04 & 3.33e-04 & 3.20e-07 & 1.94e-07\\
8.00e-01 & 7.80e-04 & 3.67e-04 & 7.81e-04 & 3.67e-04 & 4.64e-07 & 2.18e-07\\
1.00e+00 & 1.04e-03 & 3.93e-04 & 1.04e-03 & 3.94e-04 & 6.32e-07 & 2.39e-07\\
1.20e+00 & 1.32e-03 & 4.16e-04 & 1.33e-03 & 4.17e-04 & 8.28e-07 & 2.60e-07\\
1.40e+00 & 1.63e-03 & 4.38e-04 & 1.64e-03 & 4.39e-04 & 1.05e-06 & 2.82e-07\\
1.60e+00 & 1.97e-03 & 4.60e-04 & 1.97e-03 & 4.61e-04 & 1.32e-06 & 3.07e-07\\
1.80e+00 & 2.32e-03 & 4.83e-04 & 2.33e-03 & 4.84e-04 & 1.62e-06 & 3.36e-07\\
2.00e+00 & 2.69e-03 & 5.07e-04 & 2.70e-03 & 5.08e-04 & 1.96e-06 & 3.70e-07\\
\bottomrule
\end{tabular}
}
    \caption{Erros obtidos para o item $d$ com $h=0.001$}
\end{table}

\begin{figure}[H]
    \includegraphics[width=\linewidth]{results/ex1/d/h_0.001.png}
    \caption{Gráfico plotado para o item $d$ quando $h=0.001$}
\end{figure}

\begin{figure}[H]
    \includegraphics[width=\linewidth]{results/ex1/d/h_0.001_abs_error.png}
    \caption{Gráfico plotado para o erro absoluto do item $d$ quando $h=0.001$}
\end{figure}

\begin{figure}[H]
    \includegraphics[width=\linewidth]{results/ex1/d/h_0.001_rel_error.png}
    \caption{Gráfico plotado para o erro relativo do item $d$ quando $h=0.001$}
\end{figure}
\subsection{Exercício 1, item $e$}\subsubsection{$h=0.1$}

Para o item $e$, foi solicitado que se sugerisse um problema de valor inicial próprio pelo aluno, com solução analítica. Foi escolhido o problema de valor inicial dado pela Equação (\ref{eq:ivp_propria}), que descreve uma Equação Logística.

\begin{equation}\label{eq:ivp_propria}
    \begin{cases}
    y'=25\left(1 - \dfrac{y}{50}\right) \quad x \in [0, 10]\\
    y(0)=1
    \end{cases}
\end{equation}

A solução analítica para esse PVI é dada pela Equação (\ref{eq:solucao_ivp})

\begin{equation}\label{eq:solucao_ivp}
    y(x) = -49 e^{\frac{-x}{2}} + 50
\end{equation}

\begin{table}[H]
    \centering
    \resizebox{\textwidth}{!}{
\begin{tabular}{ccccc}
\toprule
x & Valor real & Valor (Euler Explícito) & Valor (Euler Implícito) & Valor (Método dos Trapézios)\\
\midrule
0.0000 & 1.0000 & 1.0000 & 1.0000 & 1.0000\\
1.0000 & 20.2800 & 20.6619 & 19.9183 & 20.2736\\
2.0000 & 31.9739 & 32.4342 & 31.5324 & 31.9661\\
3.0000 & 39.0666 & 39.4827 & 38.6625 & 39.0595\\
4.0000 & 43.3686 & 43.7029 & 43.0398 & 43.3628\\
5.0000 & 45.9778 & 46.2297 & 45.7270 & 45.9735\\
6.0000 & 47.5604 & 47.7426 & 47.3768 & 47.5573\\
7.0000 & 48.5203 & 48.6484 & 48.3896 & 48.5181\\
8.0000 & 49.1025 & 49.1907 & 49.0113 & 49.1010\\
9.0000 & 49.4557 & 49.5155 & 49.3930 & 49.4546\\
10.0000 & 49.6698 & 49.7099 & 49.6274 & 49.6691\\
\bottomrule
\end{tabular}
}
    \caption{Valores obtidos para o item $e$ com $h=0.1$}
\end{table}

\begin{table}[H]
    \centering
    \resizebox{\textwidth}{!}{
\begin{tabular}{ccccccc}
\toprule
x & Erro absoluto (Euler Explícito) & Erro relativo (Euler Explícito) & Erro absoluto (Euler Implícito) & Erro relativo (Euler Implícito) & Erro absoluto (Método dos Trapézios) & Erro relativo (Método dos Trapézios)\\
\midrule
0.00e+00 & 0.00e+00 & 0.00e+00 & 0.00e+00 & 0.00e+00 & 0.00e+00 & 0.00e+00\\
1.00e+00 & 3.82e-01 & 1.88e-02 & 3.62e-01 & 1.78e-02 & 6.43e-03 & 3.17e-04\\
2.00e+00 & 4.60e-01 & 1.44e-02 & 4.41e-01 & 1.38e-02 & 7.80e-03 & 2.44e-04\\
3.00e+00 & 4.16e-01 & 1.07e-02 & 4.04e-01 & 1.03e-02 & 7.10e-03 & 1.82e-04\\
4.00e+00 & 3.34e-01 & 7.71e-03 & 3.29e-01 & 7.58e-03 & 5.74e-03 & 1.32e-04\\
5.00e+00 & 2.52e-01 & 5.48e-03 & 2.51e-01 & 5.46e-03 & 4.35e-03 & 9.47e-05\\
6.00e+00 & 1.82e-01 & 3.83e-03 & 1.84e-01 & 3.86e-03 & 3.17e-03 & 6.66e-05\\
7.00e+00 & 1.28e-01 & 2.64e-03 & 1.31e-01 & 2.70e-03 & 2.24e-03 & 4.62e-05\\
8.00e+00 & 8.82e-02 & 1.80e-03 & 9.12e-02 & 1.86e-03 & 1.55e-03 & 3.17e-05\\
9.00e+00 & 5.98e-02 & 1.21e-03 & 6.26e-02 & 1.27e-03 & 1.06e-03 & 2.14e-05\\
1.00e+01 & 4.01e-02 & 8.06e-04 & 4.25e-02 & 8.55e-04 & 7.15e-04 & 1.44e-05\\
\bottomrule
\end{tabular}
}
    \caption{Erros obtidos para o item $e$ com $h=0.1$}
\end{table}

\begin{figure}[H]
    \includegraphics[width=\linewidth]{results/ex1/e/h_0.1.png}
    \caption{Gráfico plotado para o item $e$ quando $h=0.1$}
\end{figure}

\begin{figure}[H]
    \includegraphics[width=\linewidth]{results/ex1/e/h_0.1_abs_error.png}
    \caption{Gráfico plotado para o erro absoluto do item $e$ quando $h=0.1$}
\end{figure}

\begin{figure}[H]
    \includegraphics[width=\linewidth]{results/ex1/e/h_0.1_rel_error.png}
    \caption{Gráfico plotado para o erro relativo do item $e$ quando $h=0.1$}
\end{figure}
\subsubsection{$h=0.01$}

\begin{table}[H]
    \centering
    \resizebox{\textwidth}{!}{
\begin{tabular}{ccccc}
\toprule
x & Valor real & Valor (Euler Explícito) & Valor (Euler Implícito) & Valor (Método dos Trapézios)\\
\midrule
0.0000 & 1.0000 & 1.0000 & 1.0000 & 1.0000\\
1.0000 & 20.2800 & 20.3172 & 20.2429 & 20.2799\\
2.0000 & 31.9739 & 32.0191 & 31.9289 & 31.9738\\
3.0000 & 39.0666 & 39.1077 & 39.0257 & 39.0666\\
4.0000 & 43.3686 & 43.4018 & 43.3354 & 43.3685\\
5.0000 & 45.9778 & 46.0030 & 45.9527 & 45.9778\\
6.0000 & 47.5604 & 47.5787 & 47.5421 & 47.5604\\
7.0000 & 48.5203 & 48.5333 & 48.5074 & 48.5203\\
8.0000 & 49.1025 & 49.1115 & 49.0935 & 49.1025\\
9.0000 & 49.4557 & 49.4618 & 49.4495 & 49.4556\\
10.0000 & 49.6698 & 49.6740 & 49.6657 & 49.6698\\
\bottomrule
\end{tabular}
}
    \caption{Valores obtidos para o item $e$ com $h=0.01$}
\end{table}

\begin{table}[H]
    \centering
    \resizebox{\textwidth}{!}{
\begin{tabular}{ccccccc}
\toprule
x & Erro absoluto (Euler Explícito) & Erro relativo (Euler Explícito) & Erro absoluto (Euler Implícito) & Erro relativo (Euler Implícito) & Erro absoluto (Método dos Trapézios) & Erro relativo (Método dos Trapézios)\\
\midrule
0.00e+00 & 0.00e+00 & 0.00e+00 & 0.00e+00 & 0.00e+00 & 0.00e+00 & 0.00e+00\\
1.00e+00 & 3.73e-02 & 1.84e-03 & 3.70e-02 & 1.83e-03 & 6.21e-05 & 3.06e-06\\
2.00e+00 & 4.52e-02 & 1.41e-03 & 4.50e-02 & 1.41e-03 & 7.54e-05 & 2.36e-06\\
3.00e+00 & 4.11e-02 & 1.05e-03 & 4.09e-02 & 1.05e-03 & 6.86e-05 & 1.76e-06\\
4.00e+00 & 3.32e-02 & 7.65e-04 & 3.31e-02 & 7.64e-04 & 5.55e-05 & 1.28e-06\\
5.00e+00 & 2.51e-02 & 5.47e-04 & 2.51e-02 & 5.47e-04 & 4.21e-05 & 9.15e-07\\
6.00e+00 & 1.83e-02 & 3.85e-04 & 1.83e-02 & 3.85e-04 & 3.06e-05 & 6.44e-07\\
7.00e+00 & 1.29e-02 & 2.67e-04 & 1.30e-02 & 2.67e-04 & 2.17e-05 & 4.46e-07\\
8.00e+00 & 8.96e-03 & 1.82e-04 & 8.99e-03 & 1.83e-04 & 1.50e-05 & 3.06e-07\\
9.00e+00 & 6.11e-03 & 1.24e-04 & 6.14e-03 & 1.24e-04 & 1.02e-05 & 2.07e-07\\
1.00e+01 & 4.11e-03 & 8.28e-05 & 4.14e-03 & 8.33e-05 & 6.90e-06 & 1.39e-07\\
\bottomrule
\end{tabular}
}
    \caption{Erros obtidos para o item $e$ com $h=0.01$}
\end{table}

\begin{figure}[H]
    \includegraphics[width=\linewidth]{results/ex1/e/h_0.01.png}
    \caption{Gráfico plotado para o item $e$ quando $h=0.01$}
\end{figure}

\begin{figure}[H]
    \includegraphics[width=\linewidth]{results/ex1/e/h_0.01_abs_error.png}
    \caption{Gráfico plotado para o erro absoluto do item $e$ quando $h=0.01$}
\end{figure}

\begin{figure}[H]
    \includegraphics[width=\linewidth]{results/ex1/e/h_0.01_rel_error.png}
    \caption{Gráfico plotado para o erro relativo do item $e$ quando $h=0.01$}
\end{figure}
\subsubsection{$h=0.005$}

\begin{table}[H]
    \centering
    \resizebox{\textwidth}{!}{
\begin{tabular}{ccccc}
\toprule
x & Valor real & Valor (Euler Explícito) & Valor (Euler Implícito) & Valor (Método dos Trapézios)\\
\midrule
0.0000 & 1.0000 & 1.0000 & 1.0000 & 1.0000\\
1.0000 & 20.2800 & 20.2986 & 20.2614 & 20.2800\\
2.0000 & 31.9739 & 31.9965 & 31.9514 & 31.9739\\
3.0000 & 39.0666 & 39.0871 & 39.0461 & 39.0666\\
4.0000 & 43.3686 & 43.3852 & 43.3520 & 43.3686\\
5.0000 & 45.9778 & 45.9904 & 45.9653 & 45.9778\\
6.0000 & 47.5604 & 47.5696 & 47.5513 & 47.5604\\
7.0000 & 48.5203 & 48.5268 & 48.5139 & 48.5203\\
8.0000 & 49.1025 & 49.1070 & 49.0980 & 49.1025\\
9.0000 & 49.4557 & 49.4587 & 49.4526 & 49.4557\\
10.0000 & 49.6698 & 49.6719 & 49.6678 & 49.6698\\
\bottomrule
\end{tabular}
}
    \caption{Valores obtidos para o item $e$ com $h=0.005$}
\end{table}

\begin{table}[H]
    \centering
    \resizebox{\textwidth}{!}{
\begin{tabular}{ccccccc}
\toprule
x & Erro absoluto (Euler Explícito) & Erro relativo (Euler Explícito) & Erro absoluto (Euler Implícito) & Erro relativo (Euler Implícito) & Erro absoluto (Método dos Trapézios) & Erro relativo (Método dos Trapézios)\\
\midrule
0.00e+00 & 0.00e+00 & 0.00e+00 & 0.00e+00 & 0.00e+00 & 0.00e+00 & 0.00e+00\\
1.00e+00 & 1.86e-02 & 9.17e-04 & 1.85e-02 & 9.15e-04 & 1.55e-05 & 7.65e-07\\
2.00e+00 & 2.26e-02 & 7.05e-04 & 2.25e-02 & 7.04e-04 & 1.88e-05 & 5.88e-07\\
3.00e+00 & 2.05e-02 & 5.25e-04 & 2.05e-02 & 5.24e-04 & 1.71e-05 & 4.38e-07\\
4.00e+00 & 1.66e-02 & 3.82e-04 & 1.66e-02 & 3.82e-04 & 1.38e-05 & 3.19e-07\\
5.00e+00 & 1.26e-02 & 2.73e-04 & 1.26e-02 & 2.73e-04 & 1.05e-05 & 2.28e-07\\
6.00e+00 & 9.15e-03 & 1.92e-04 & 9.15e-03 & 1.92e-04 & 7.64e-06 & 1.61e-07\\
7.00e+00 & 6.47e-03 & 1.33e-04 & 6.48e-03 & 1.33e-04 & 5.40e-06 & 1.11e-07\\
8.00e+00 & 4.48e-03 & 9.13e-05 & 4.49e-03 & 9.15e-05 & 3.75e-06 & 7.63e-08\\
9.00e+00 & 3.06e-03 & 6.18e-05 & 3.07e-03 & 6.20e-05 & 2.56e-06 & 5.17e-08\\
1.00e+01 & 2.06e-03 & 4.15e-05 & 2.07e-03 & 4.16e-05 & 1.72e-06 & 3.47e-08\\
\bottomrule
\end{tabular}
}
    \caption{Erros obtidos para o item $e$ com $h=0.005$}
\end{table}

\begin{figure}[H]
    \includegraphics[width=\linewidth]{results/ex1/e/h_0.005.png}
    \caption{Gráfico plotado para o item $e$ quando $h=0.005$}
\end{figure}

\begin{figure}[H]
    \includegraphics[width=\linewidth]{results/ex1/e/h_0.005_abs_error.png}
    \caption{Gráfico plotado para o erro absoluto do item $e$ quando $h=0.005$}
\end{figure}

\begin{figure}[H]
    \includegraphics[width=\linewidth]{results/ex1/e/h_0.005_rel_error.png}
    \caption{Gráfico plotado para o erro relativo do item $e$ quando $h=0.005$}
\end{figure}
\subsubsection{$h=0.001$}

\begin{table}[H]
    \centering
    \resizebox{\textwidth}{!}{
\begin{tabular}{ccccc}
\toprule
x & Valor real & Valor (Euler Explícito) & Valor (Euler Implícito) & Valor (Método dos Trapézios)\\
\midrule
0.0000 & 1.0000 & 1.0000 & 1.0000 & 1.0000\\
1.0000 & 20.2800 & 20.2837 & 20.2763 & 20.2800\\
2.0000 & 31.9739 & 31.9784 & 31.9694 & 31.9739\\
3.0000 & 39.0666 & 39.0707 & 39.0625 & 39.0666\\
4.0000 & 43.3686 & 43.3719 & 43.3653 & 43.3686\\
5.0000 & 45.9778 & 45.9803 & 45.9753 & 45.9778\\
6.0000 & 47.5604 & 47.5623 & 47.5586 & 47.5604\\
7.0000 & 48.5203 & 48.5216 & 48.5190 & 48.5203\\
8.0000 & 49.1025 & 49.1034 & 49.1016 & 49.1025\\
9.0000 & 49.4557 & 49.4563 & 49.4550 & 49.4557\\
10.0000 & 49.6698 & 49.6703 & 49.6694 & 49.6698\\
\bottomrule
\end{tabular}
}
    \caption{Valores obtidos para o item $e$ com $h=0.001$}
\end{table}

\begin{table}[H]
    \centering
    \resizebox{\textwidth}{!}{
\begin{tabular}{ccccccc}
\toprule
x & Erro absoluto (Euler Explícito) & Erro relativo (Euler Explícito) & Erro absoluto (Euler Implícito) & Erro relativo (Euler Implícito) & Erro absoluto (Método dos Trapézios) & Erro relativo (Método dos Trapézios)\\
\midrule
0.00e+00 & 0.00e+00 & 0.00e+00 & 0.00e+00 & 0.00e+00 & 0.00e+00 & 0.00e+00\\
1.00e+00 & 3.72e-03 & 1.83e-04 & 3.71e-03 & 1.83e-04 & 6.19e-07 & 3.05e-08\\
2.00e+00 & 4.51e-03 & 1.41e-04 & 4.51e-03 & 1.41e-04 & 7.51e-07 & 2.35e-08\\
3.00e+00 & 4.10e-03 & 1.05e-04 & 4.10e-03 & 1.05e-04 & 6.84e-07 & 1.75e-08\\
4.00e+00 & 3.32e-03 & 7.65e-05 & 3.32e-03 & 7.64e-05 & 5.53e-07 & 1.27e-08\\
5.00e+00 & 2.51e-03 & 5.47e-05 & 2.51e-03 & 5.47e-05 & 4.19e-07 & 9.12e-09\\
6.00e+00 & 1.83e-03 & 3.85e-05 & 1.83e-03 & 3.85e-05 & 3.05e-07 & 6.41e-09\\
7.00e+00 & 1.29e-03 & 2.67e-05 & 1.29e-03 & 2.67e-05 & 2.16e-07 & 4.45e-09\\
8.00e+00 & 8.97e-04 & 1.83e-05 & 8.98e-04 & 1.83e-05 & 1.50e-07 & 3.05e-09\\
9.00e+00 & 6.12e-04 & 1.24e-05 & 6.13e-04 & 1.24e-05 & 1.02e-07 & 2.06e-09\\
1.00e+01 & 4.13e-04 & 8.31e-06 & 4.13e-04 & 8.31e-06 & 6.88e-08 & 1.39e-09\\
\bottomrule
\end{tabular}
}
    \caption{Erros obtidos para o item $e$ com $h=0.001$}
\end{table}

\begin{figure}[H]
    \includegraphics[width=\linewidth]{results/ex1/e/h_0.001.png}
    \caption{Gráfico plotado para o item $e$ quando $h=0.001$}
\end{figure}

\begin{figure}[H]
    \includegraphics[width=\linewidth]{results/ex1/e/h_0.001_abs_error.png}
    \caption{Gráfico plotado para o erro absoluto do item $e$ quando $h=0.001$}
\end{figure}

\begin{figure}[H]
    \includegraphics[width=\linewidth]{results/ex1/e/h_0.001_rel_error.png}
    \caption{Gráfico plotado para o erro relativo do item $e$ quando $h=0.001$}
\end{figure}
\subsection{Exercício 2, item $a$}\subsubsection{$h=0.1$}

\begin{table}[H]
    \centering
    \resizebox{\textwidth}{!}{
\begin{tabular}{cccc}
\toprule
x & Valor real & Valor (Runge-Kutta de 3a ordem) & Valor (Runge-Kutta de 4a ordem)\\
\midrule
0.0000 & 2.0000 & 2.0000 & 2.0000\\
0.1000 & 2.0048 & 2.0048 & 2.0048\\
0.2000 & 2.0187 & 2.0187 & 2.0187\\
0.3000 & 2.0408 & 2.0408 & 2.0408\\
0.4000 & 2.0703 & 2.0703 & 2.0703\\
0.5000 & 2.1065 & 2.1065 & 2.1065\\
0.6000 & 2.1488 & 2.1488 & 2.1488\\
0.7000 & 2.1966 & 2.1966 & 2.1966\\
0.8000 & 2.2493 & 2.2493 & 2.2493\\
0.9000 & 2.3066 & 2.3066 & 2.3066\\
1.0000 & 2.3679 & 2.3679 & 2.3679\\
\bottomrule
\end{tabular}
}
    \caption{Valores obtidos para o item $a$ com $h=0.1$}
\end{table}

\begin{table}[H]
    \centering
    \resizebox{\textwidth}{!}{
\begin{tabular}{ccccc}
\toprule
x & Erro absoluto (Runge-Kutta de 3a ordem) & Erro relativo (Runge-Kutta de 3a ordem) & Erro absoluto (Runge-Kutta de 4a ordem) & Erro relativo (Runge-Kutta de 4a ordem)\\
\midrule
0.00e+00 & 0.00e+00 & 0.00e+00 & 0.00e+00 & 0.00e+00\\
1.00e-01 & 4.08e-06 & 2.04e-06 & 8.20e-08 & 4.09e-08\\
2.00e-01 & 7.39e-06 & 3.66e-06 & 1.48e-07 & 7.35e-08\\
3.00e-01 & 1.00e-05 & 4.92e-06 & 2.01e-07 & 9.86e-08\\
4.00e-01 & 1.21e-05 & 5.85e-06 & 2.43e-07 & 1.17e-07\\
5.00e-01 & 1.37e-05 & 6.50e-06 & 2.75e-07 & 1.30e-07\\
6.00e-01 & 1.49e-05 & 6.92e-06 & 2.98e-07 & 1.39e-07\\
7.00e-01 & 1.57e-05 & 7.14e-06 & 3.15e-07 & 1.43e-07\\
8.00e-01 & 1.62e-05 & 7.21e-06 & 3.26e-07 & 1.45e-07\\
9.00e-01 & 1.65e-05 & 7.16e-06 & 3.31e-07 & 1.44e-07\\
1.00e+00 & 1.66e-05 & 7.01e-06 & 3.33e-07 & 1.41e-07\\
\bottomrule
\end{tabular}
}
    \caption{Erros obtidos para o item $a$ com $h=0.1$}
\end{table}

\begin{figure}[H]
    \includegraphics[width=\linewidth]{results/ex2/a/h_0.1.png}
    \caption{Gráfico plotado para o item $a$ quando $h=0.1$}
\end{figure}

\begin{figure}[H]
    \includegraphics[width=\linewidth]{results/ex2/a/h_0.1_abs_error.png}
    \caption{Gráfico plotado para o erro absoluto do item $a$ quando $h=0.1$}
\end{figure}

\begin{figure}[H]
    \includegraphics[width=\linewidth]{results/ex2/a/h_0.1_rel_error.png}
    \caption{Gráfico plotado para o erro relativo do item $a$ quando $h=0.1$}
\end{figure}
\subsubsection{$h=0.01$}

\begin{table}[H]
    \centering
    \resizebox{\textwidth}{!}{
\begin{tabular}{cccc}
\toprule
x & Valor real & Valor (Runge-Kutta de 3a ordem) & Valor (Runge-Kutta de 4a ordem)\\
\midrule
0.0000 & 2.0000 & 2.0000 & 2.0000\\
0.1000 & 2.0048 & 2.0048 & 2.0048\\
0.2000 & 2.0187 & 2.0187 & 2.0187\\
0.3000 & 2.0408 & 2.0408 & 2.0408\\
0.4000 & 2.0703 & 2.0703 & 2.0703\\
0.5000 & 2.1065 & 2.1065 & 2.1065\\
0.6000 & 2.1488 & 2.1488 & 2.1488\\
0.7000 & 2.1966 & 2.1966 & 2.1966\\
0.8000 & 2.2493 & 2.2493 & 2.2493\\
0.9000 & 2.3066 & 2.3066 & 2.3066\\
1.0000 & 2.3679 & 2.3679 & 2.3679\\
\bottomrule
\end{tabular}
}
    \caption{Valores obtidos para o item $a$ com $h=0.01$}
\end{table}

\begin{table}[H]
    \centering
    \resizebox{\textwidth}{!}{
\begin{tabular}{ccccc}
\toprule
x & Erro absoluto (Runge-Kutta de 3a ordem) & Erro relativo (Runge-Kutta de 3a ordem) & Erro absoluto (Runge-Kutta de 4a ordem) & Erro relativo (Runge-Kutta de 4a ordem)\\
\midrule
0.00e+00 & 0.00e+00 & 0.00e+00 & 0.00e+00 & 0.00e+00\\
1.00e-01 & 3.80e-09 & 1.90e-09 & 7.60e-12 & 3.79e-12\\
2.00e-01 & 6.88e-09 & 3.41e-09 & 1.38e-11 & 6.82e-12\\
3.00e-01 & 9.33e-09 & 4.57e-09 & 1.87e-11 & 9.15e-12\\
4.00e-01 & 1.13e-08 & 5.44e-09 & 2.25e-11 & 1.09e-11\\
5.00e-01 & 1.27e-08 & 6.05e-09 & 2.55e-11 & 1.21e-11\\
6.00e-01 & 1.38e-08 & 6.44e-09 & 2.77e-11 & 1.29e-11\\
7.00e-01 & 1.46e-08 & 6.65e-09 & 2.92e-11 & 1.33e-11\\
8.00e-01 & 1.51e-08 & 6.71e-09 & 3.02e-11 & 1.34e-11\\
9.00e-01 & 1.54e-08 & 6.66e-09 & 3.07e-11 & 1.33e-11\\
1.00e+00 & 1.55e-08 & 6.53e-09 & 3.09e-11 & 1.31e-11\\
\bottomrule
\end{tabular}
}
    \caption{Erros obtidos para o item $a$ com $h=0.01$}
\end{table}

\begin{figure}[H]
    \includegraphics[width=\linewidth]{results/ex2/a/h_0.01.png}
    \caption{Gráfico plotado para o item $a$ quando $h=0.01$}
\end{figure}

\begin{figure}[H]
    \includegraphics[width=\linewidth]{results/ex2/a/h_0.01_abs_error.png}
    \caption{Gráfico plotado para o erro absoluto do item $a$ quando $h=0.01$}
\end{figure}

\begin{figure}[H]
    \includegraphics[width=\linewidth]{results/ex2/a/h_0.01_rel_error.png}
    \caption{Gráfico plotado para o erro relativo do item $a$ quando $h=0.01$}
\end{figure}
\subsubsection{$h=0.005$}

\begin{table}[H]
    \centering
    \resizebox{\textwidth}{!}{
\begin{tabular}{cccc}
\toprule
x & Valor real & Valor (Runge-Kutta de 3a ordem) & Valor (Runge-Kutta de 4a ordem)\\
\midrule
0.0000 & 2.0000 & 2.0000 & 2.0000\\
0.1000 & 2.0048 & 2.0048 & 2.0048\\
0.2000 & 2.0187 & 2.0187 & 2.0187\\
0.3000 & 2.0408 & 2.0408 & 2.0408\\
0.4000 & 2.0703 & 2.0703 & 2.0703\\
0.5000 & 2.1065 & 2.1065 & 2.1065\\
0.6000 & 2.1488 & 2.1488 & 2.1488\\
0.7000 & 2.1966 & 2.1966 & 2.1966\\
0.8000 & 2.2493 & 2.2493 & 2.2493\\
0.9000 & 2.3066 & 2.3066 & 2.3066\\
1.0000 & 2.3679 & 2.3679 & 2.3679\\
\bottomrule
\end{tabular}
}
    \caption{Valores obtidos para o item $a$ com $h=0.005$}
\end{table}

\begin{table}[H]
    \centering
    \resizebox{\textwidth}{!}{
\begin{tabular}{ccccc}
\toprule
x & Erro absoluto (Runge-Kutta de 3a ordem) & Erro relativo (Runge-Kutta de 3a ordem) & Erro absoluto (Runge-Kutta de 4a ordem) & Erro relativo (Runge-Kutta de 4a ordem)\\
\midrule
0.00e+00 & 0.00e+00 & 0.00e+00 & 0.00e+00 & 0.00e+00\\
1.00e-01 & 4.73e-10 & 2.36e-10 & 4.73e-13 & 2.36e-13\\
2.00e-01 & 8.56e-10 & 4.24e-10 & 8.56e-13 & 4.24e-13\\
3.00e-01 & 1.16e-09 & 5.69e-10 & 1.16e-12 & 5.70e-13\\
4.00e-01 & 1.40e-09 & 6.77e-10 & 1.40e-12 & 6.77e-13\\
5.00e-01 & 1.59e-09 & 7.53e-10 & 1.59e-12 & 7.53e-13\\
6.00e-01 & 1.72e-09 & 8.01e-10 & 1.72e-12 & 8.02e-13\\
7.00e-01 & 1.82e-09 & 8.28e-10 & 1.82e-12 & 8.28e-13\\
8.00e-01 & 1.88e-09 & 8.36e-10 & 1.88e-12 & 8.36e-13\\
9.00e-01 & 1.91e-09 & 8.30e-10 & 1.91e-12 & 8.29e-13\\
1.00e+00 & 1.92e-09 & 8.12e-10 & 1.92e-12 & 8.13e-13\\
\bottomrule
\end{tabular}
}
    \caption{Erros obtidos para o item $a$ com $h=0.005$}
\end{table}

\begin{figure}[H]
    \includegraphics[width=\linewidth]{results/ex2/a/h_0.005.png}
    \caption{Gráfico plotado para o item $a$ quando $h=0.005$}
\end{figure}

\begin{figure}[H]
    \includegraphics[width=\linewidth]{results/ex2/a/h_0.005_abs_error.png}
    \caption{Gráfico plotado para o erro absoluto do item $a$ quando $h=0.005$}
\end{figure}

\begin{figure}[H]
    \includegraphics[width=\linewidth]{results/ex2/a/h_0.005_rel_error.png}
    \caption{Gráfico plotado para o erro relativo do item $a$ quando $h=0.005$}
\end{figure}
\subsubsection{$h=0.001$}

\begin{table}[H]
    \centering
    \resizebox{\textwidth}{!}{
\begin{tabular}{cccc}
\toprule
x & Valor real & Valor (Runge-Kutta de 3a ordem) & Valor (Runge-Kutta de 4a ordem)\\
\midrule
0.0000 & 2.0000 & 2.0000 & 2.0000\\
0.1000 & 2.0048 & 2.0048 & 2.0048\\
0.2000 & 2.0187 & 2.0187 & 2.0187\\
0.3000 & 2.0408 & 2.0408 & 2.0408\\
0.4000 & 2.0703 & 2.0703 & 2.0703\\
0.5000 & 2.1065 & 2.1065 & 2.1065\\
0.6000 & 2.1488 & 2.1488 & 2.1488\\
0.7000 & 2.1966 & 2.1966 & 2.1966\\
0.8000 & 2.2493 & 2.2493 & 2.2493\\
0.9000 & 2.3066 & 2.3066 & 2.3066\\
1.0000 & 2.3679 & 2.3679 & 2.3679\\
\bottomrule
\end{tabular}
}
    \caption{Valores obtidos para o item $a$ com $h=0.001$}
\end{table}

\begin{table}[H]
    \centering
    \resizebox{\textwidth}{!}{
\begin{tabular}{ccccc}
\toprule
x & Erro absoluto (Runge-Kutta de 3a ordem) & Erro relativo (Runge-Kutta de 3a ordem) & Erro absoluto (Runge-Kutta de 4a ordem) & Erro relativo (Runge-Kutta de 4a ordem)\\
\midrule
0.00e+00 & 0.00e+00 & 0.00e+00 & 0.00e+00 & 0.00e+00\\
1.00e-01 & 3.77e-12 & 1.88e-12 & 4.44e-16 & 2.22e-16\\
2.00e-01 & 6.83e-12 & 3.38e-12 & 8.88e-16 & 4.40e-16\\
3.00e-01 & 9.27e-12 & 4.54e-12 & 1.33e-15 & 6.53e-16\\
4.00e-01 & 1.12e-11 & 5.40e-12 & 1.33e-15 & 6.44e-16\\
5.00e-01 & 1.26e-11 & 6.00e-12 & 1.33e-15 & 6.32e-16\\
6.00e-01 & 1.37e-11 & 6.39e-12 & 2.22e-15 & 1.03e-15\\
7.00e-01 & 1.45e-11 & 6.60e-12 & 4.44e-16 & 2.02e-16\\
8.00e-01 & 1.50e-11 & 6.67e-12 & 4.44e-16 & 1.97e-16\\
9.00e-01 & 1.53e-11 & 6.62e-12 & 4.44e-16 & 1.93e-16\\
1.00e+00 & 1.53e-11 & 6.48e-12 & 4.44e-16 & 1.88e-16\\
\bottomrule
\end{tabular}
}
    \caption{Erros obtidos para o item $a$ com $h=0.001$}
\end{table}

\begin{figure}[H]
    \includegraphics[width=\linewidth]{results/ex2/a/h_0.001.png}
    \caption{Gráfico plotado para o item $a$ quando $h=0.001$}
\end{figure}

\begin{figure}[H]
    \includegraphics[width=\linewidth]{results/ex2/a/h_0.001_abs_error.png}
    \caption{Gráfico plotado para o erro absoluto do item $a$ quando $h=0.001$}
\end{figure}

\begin{figure}[H]
    \includegraphics[width=\linewidth]{results/ex2/a/h_0.001_rel_error.png}
    \caption{Gráfico plotado para o erro relativo do item $a$ quando $h=0.001$}
\end{figure}
\subsection{Exercício 2, item $b$}\subsubsection{$h=0.1$}

\begin{table}[H]
    \centering
    \resizebox{\textwidth}{!}{
\begin{tabular}{cccc}
\toprule
x & Valor real & Valor (Runge-Kutta de 3a ordem) & Valor (Runge-Kutta de 4a ordem)\\
\midrule
0.0000 & 0.5000 & 0.5000 & 0.5000\\
0.1000 & 0.4975 & 0.4975 & 0.4975\\
0.2000 & 0.4902 & 0.4902 & 0.4902\\
0.3000 & 0.4785 & 0.4785 & 0.4785\\
0.4000 & 0.4630 & 0.4630 & 0.4630\\
0.5000 & 0.4444 & 0.4444 & 0.4444\\
0.6000 & 0.4237 & 0.4237 & 0.4237\\
0.7000 & 0.4016 & 0.4016 & 0.4016\\
0.8000 & 0.3788 & 0.3788 & 0.3788\\
0.9000 & 0.3559 & 0.3559 & 0.3559\\
1.0000 & 0.3333 & 0.3333 & 0.3333\\
\bottomrule
\end{tabular}
}
    \caption{Valores obtidos para o item $b$ com $h=0.1$}
\end{table}

\begin{table}[H]
    \centering
    \resizebox{\textwidth}{!}{
\begin{tabular}{ccccc}
\toprule
x & Erro absoluto (Runge-Kutta de 3a ordem) & Erro relativo (Runge-Kutta de 3a ordem) & Erro absoluto (Runge-Kutta de 4a ordem) & Erro relativo (Runge-Kutta de 4a ordem)\\
\midrule
0.00e+00 & 0.00e+00 & 0.00e+00 & 0.00e+00 & 0.00e+00\\
1.00e-01 & 3.88e-08 & 7.79e-08 & 5.26e-09 & 1.06e-08\\
2.00e-01 & 2.78e-07 & 5.66e-07 & 2.04e-08 & 4.16e-08\\
3.00e-01 & 8.03e-07 & 1.68e-06 & 4.20e-08 & 8.78e-08\\
4.00e-01 & 1.57e-06 & 3.40e-06 & 6.47e-08 & 1.40e-07\\
5.00e-01 & 2.44e-06 & 5.49e-06 & 8.26e-08 & 1.86e-07\\
6.00e-01 & 3.21e-06 & 7.57e-06 & 9.14e-08 & 2.16e-07\\
7.00e-01 & 3.71e-06 & 9.23e-06 & 8.89e-08 & 2.21e-07\\
8.00e-01 & 3.84e-06 & 1.01e-05 & 7.56e-08 & 2.00e-07\\
9.00e-01 & 3.59e-06 & 1.01e-05 & 5.38e-08 & 1.51e-07\\
1.00e+00 & 3.02e-06 & 9.06e-06 & 2.68e-08 & 8.04e-08\\
\bottomrule
\end{tabular}
}
    \caption{Erros obtidos para o item $b$ com $h=0.1$}
\end{table}

\begin{figure}[H]
    \includegraphics[width=\linewidth]{results/ex2/b/h_0.1.png}
    \caption{Gráfico plotado para o item $b$ quando $h=0.1$}
\end{figure}

\begin{figure}[H]
    \includegraphics[width=\linewidth]{results/ex2/b/h_0.1_abs_error.png}
    \caption{Gráfico plotado para o erro absoluto do item $b$ quando $h=0.1$}
\end{figure}

\begin{figure}[H]
    \includegraphics[width=\linewidth]{results/ex2/b/h_0.1_rel_error.png}
    \caption{Gráfico plotado para o erro relativo do item $b$ quando $h=0.1$}
\end{figure}
\subsubsection{$h=0.01$}

\begin{table}[H]
    \centering
    \resizebox{\textwidth}{!}{
\begin{tabular}{cccc}
\toprule
x & Valor real & Valor (Runge-Kutta de 3a ordem) & Valor (Runge-Kutta de 4a ordem)\\
\midrule
0.0000 & 0.5000 & 0.5000 & 0.5000\\
0.1000 & 0.4975 & 0.4975 & 0.4975\\
0.2000 & 0.4902 & 0.4902 & 0.4902\\
0.3000 & 0.4785 & 0.4785 & 0.4785\\
0.4000 & 0.4630 & 0.4630 & 0.4630\\
0.5000 & 0.4444 & 0.4444 & 0.4444\\
0.6000 & 0.4237 & 0.4237 & 0.4237\\
0.7000 & 0.4016 & 0.4016 & 0.4016\\
0.8000 & 0.3788 & 0.3788 & 0.3788\\
0.9000 & 0.3559 & 0.3559 & 0.3559\\
1.0000 & 0.3333 & 0.3333 & 0.3333\\
\bottomrule
\end{tabular}
}
    \caption{Valores obtidos para o item $b$ com $h=0.01$}
\end{table}

\begin{table}[H]
    \centering
    \resizebox{\textwidth}{!}{
\begin{tabular}{ccccc}
\toprule
x & Erro absoluto (Runge-Kutta de 3a ordem) & Erro relativo (Runge-Kutta de 3a ordem) & Erro absoluto (Runge-Kutta de 4a ordem) & Erro relativo (Runge-Kutta de 4a ordem)\\
\midrule
0.00e+00 & 0.00e+00 & 0.00e+00 & 0.00e+00 & 0.00e+00\\
1.00e-01 & 3.19e-11 & 6.42e-11 & 5.10e-13 & 1.03e-12\\
2.00e-01 & 2.31e-10 & 4.72e-10 & 1.90e-12 & 3.87e-12\\
3.00e-01 & 6.80e-10 & 1.42e-09 & 3.78e-12 & 7.90e-12\\
4.00e-01 & 1.34e-09 & 2.90e-09 & 5.63e-12 & 1.22e-11\\
5.00e-01 & 2.08e-09 & 4.68e-09 & 6.95e-12 & 1.56e-11\\
6.00e-01 & 2.72e-09 & 6.41e-09 & 7.40e-12 & 1.75e-11\\
7.00e-01 & 3.11e-09 & 7.75e-09 & 6.86e-12 & 1.71e-11\\
8.00e-01 & 3.18e-09 & 8.40e-09 & 5.41e-12 & 1.43e-11\\
9.00e-01 & 2.92e-09 & 8.20e-09 & 3.28e-12 & 9.22e-12\\
1.00e+00 & 2.38e-09 & 7.13e-09 & 7.61e-13 & 2.28e-12\\
\bottomrule
\end{tabular}
}
    \caption{Erros obtidos para o item $b$ com $h=0.01$}
\end{table}

\begin{figure}[H]
    \includegraphics[width=\linewidth]{results/ex2/b/h_0.01.png}
    \caption{Gráfico plotado para o item $b$ quando $h=0.01$}
\end{figure}

\begin{figure}[H]
    \includegraphics[width=\linewidth]{results/ex2/b/h_0.01_abs_error.png}
    \caption{Gráfico plotado para o erro absoluto do item $b$ quando $h=0.01$}
\end{figure}

\begin{figure}[H]
    \includegraphics[width=\linewidth]{results/ex2/b/h_0.01_rel_error.png}
    \caption{Gráfico plotado para o erro relativo do item $b$ quando $h=0.01$}
\end{figure}
\subsubsection{$h=0.005$}

\begin{table}[H]
    \centering
    \resizebox{\textwidth}{!}{
\begin{tabular}{cccc}
\toprule
x & Valor real & Valor (Runge-Kutta de 3a ordem) & Valor (Runge-Kutta de 4a ordem)\\
\midrule
0.0000 & 0.5000 & 0.5000 & 0.5000\\
0.1000 & 0.4975 & 0.4975 & 0.4975\\
0.2000 & 0.4902 & 0.4902 & 0.4902\\
0.3000 & 0.4785 & 0.4785 & 0.4785\\
0.4000 & 0.4630 & 0.4630 & 0.4630\\
0.5000 & 0.4444 & 0.4444 & 0.4444\\
0.6000 & 0.4237 & 0.4237 & 0.4237\\
0.7000 & 0.4016 & 0.4016 & 0.4016\\
0.8000 & 0.3788 & 0.3788 & 0.3788\\
0.9000 & 0.3559 & 0.3559 & 0.3559\\
1.0000 & 0.3333 & 0.3333 & 0.3333\\
\bottomrule
\end{tabular}
}
    \caption{Valores obtidos para o item $b$ com $h=0.005$}
\end{table}

\begin{table}[H]
    \centering
    \resizebox{\textwidth}{!}{
\begin{tabular}{ccccc}
\toprule
x & Erro absoluto (Runge-Kutta de 3a ordem) & Erro relativo (Runge-Kutta de 3a ordem) & Erro absoluto (Runge-Kutta de 4a ordem) & Erro relativo (Runge-Kutta de 4a ordem)\\
\midrule
0.00e+00 & 0.00e+00 & 0.00e+00 & 0.00e+00 & 0.00e+00\\
1.00e-01 & 3.90e-12 & 7.84e-12 & 3.18e-14 & 6.39e-14\\
2.00e-01 & 2.85e-11 & 5.81e-11 & 1.18e-13 & 2.41e-13\\
3.00e-01 & 8.41e-11 & 1.76e-10 & 2.34e-13 & 4.90e-13\\
4.00e-01 & 1.66e-10 & 3.59e-10 & 3.48e-13 & 7.53e-13\\
5.00e-01 & 2.57e-10 & 5.79e-10 & 4.30e-13 & 9.67e-13\\
6.00e-01 & 3.36e-10 & 7.94e-10 & 4.57e-13 & 1.08e-12\\
7.00e-01 & 3.85e-10 & 9.59e-10 & 4.22e-13 & 1.05e-12\\
8.00e-01 & 3.94e-10 & 1.04e-09 & 3.31e-13 & 8.74e-13\\
9.00e-01 & 3.61e-10 & 1.01e-09 & 1.98e-13 & 5.57e-13\\
1.00e+00 & 2.93e-10 & 8.79e-10 & 4.15e-14 & 1.24e-13\\
\bottomrule
\end{tabular}
}
    \caption{Erros obtidos para o item $b$ com $h=0.005$}
\end{table}

\begin{figure}[H]
    \includegraphics[width=\linewidth]{results/ex2/b/h_0.005.png}
    \caption{Gráfico plotado para o item $b$ quando $h=0.005$}
\end{figure}

\begin{figure}[H]
    \includegraphics[width=\linewidth]{results/ex2/b/h_0.005_abs_error.png}
    \caption{Gráfico plotado para o erro absoluto do item $b$ quando $h=0.005$}
\end{figure}

\begin{figure}[H]
    \includegraphics[width=\linewidth]{results/ex2/b/h_0.005_rel_error.png}
    \caption{Gráfico plotado para o erro relativo do item $b$ quando $h=0.005$}
\end{figure}
\subsubsection{$h=0.001$}

\begin{table}[H]
    \centering
    \resizebox{\textwidth}{!}{
\begin{tabular}{cccc}
\toprule
x & Valor real & Valor (Runge-Kutta de 3a ordem) & Valor (Runge-Kutta de 4a ordem)\\
\midrule
0.0000 & 0.5000 & 0.5000 & 0.5000\\
0.1000 & 0.4975 & 0.4975 & 0.4975\\
0.2000 & 0.4902 & 0.4902 & 0.4902\\
0.3000 & 0.4785 & 0.4785 & 0.4785\\
0.4000 & 0.4630 & 0.4630 & 0.4630\\
0.5000 & 0.4444 & 0.4444 & 0.4444\\
0.6000 & 0.4237 & 0.4237 & 0.4237\\
0.7000 & 0.4016 & 0.4016 & 0.4016\\
0.8000 & 0.3788 & 0.3788 & 0.3788\\
0.9000 & 0.3559 & 0.3559 & 0.3559\\
1.0000 & 0.3333 & 0.3333 & 0.3333\\
\bottomrule
\end{tabular}
}
    \caption{Valores obtidos para o item $b$ com $h=0.001$}
\end{table}

\begin{table}[H]
    \centering
    \resizebox{\textwidth}{!}{
\begin{tabular}{ccccc}
\toprule
x & Erro absoluto (Runge-Kutta de 3a ordem) & Erro relativo (Runge-Kutta de 3a ordem) & Erro absoluto (Runge-Kutta de 4a ordem) & Erro relativo (Runge-Kutta de 4a ordem)\\
\midrule
0.00e+00 & 0.00e+00 & 0.00e+00 & 0.00e+00 & 0.00e+00\\
1.00e-01 & 3.08e-14 & 6.19e-14 & 5.55e-16 & 1.12e-15\\
2.00e-01 & 2.26e-13 & 4.61e-13 & 5.55e-16 & 1.13e-15\\
3.00e-01 & 6.67e-13 & 1.40e-12 & 5.55e-16 & 1.16e-15\\
4.00e-01 & 1.32e-12 & 2.85e-12 & 5.55e-16 & 1.20e-15\\
5.00e-01 & 2.04e-12 & 4.60e-12 & 7.22e-16 & 1.62e-15\\
6.00e-01 & 2.67e-12 & 6.31e-12 & 9.44e-16 & 2.23e-15\\
7.00e-01 & 3.06e-12 & 7.61e-12 & 6.11e-16 & 1.52e-15\\
8.00e-01 & 3.12e-12 & 8.24e-12 & 5.00e-16 & 1.32e-15\\
9.00e-01 & 2.86e-12 & 8.03e-12 & 1.11e-16 & 3.12e-16\\
1.00e+00 & 2.32e-12 & 6.95e-12 & 0.00e+00 & 0.00e+00\\
\bottomrule
\end{tabular}
}
    \caption{Erros obtidos para o item $b$ com $h=0.001$}
\end{table}

\begin{figure}[H]
    \includegraphics[width=\linewidth]{results/ex2/b/h_0.001.png}
    \caption{Gráfico plotado para o item $b$ quando $h=0.001$}
\end{figure}

\begin{figure}[H]
    \includegraphics[width=\linewidth]{results/ex2/b/h_0.001_abs_error.png}
    \caption{Gráfico plotado para o erro absoluto do item $b$ quando $h=0.001$}
\end{figure}

\begin{figure}[H]
    \includegraphics[width=\linewidth]{results/ex2/b/h_0.001_rel_error.png}
    \caption{Gráfico plotado para o erro relativo do item $b$ quando $h=0.001$}
\end{figure}
\subsection{Exercício 3, item $a$}\subsubsection{$h=0.1$}

\begin{table}[H]
    \centering
    \resizebox{\textwidth}{!}{
\begin{tabular}{cccc}
\toprule
x & Valor real & Valor (Euler Modificado) & Valor (Euler Aperfeiçoado)\\
\midrule
0.0000 & 2.0000 & 2.0000 & 2.0000\\
0.1000 & 2.0048 & 2.0050 & 2.0050\\
0.2000 & 2.0187 & 2.0190 & 2.0190\\
0.3000 & 2.0408 & 2.0412 & 2.0412\\
0.4000 & 2.0703 & 2.0708 & 2.0708\\
0.5000 & 2.1065 & 2.1071 & 2.1071\\
0.6000 & 2.1488 & 2.1494 & 2.1494\\
0.7000 & 2.1966 & 2.1972 & 2.1972\\
0.8000 & 2.2493 & 2.2500 & 2.2500\\
0.9000 & 2.3066 & 2.3072 & 2.3072\\
1.0000 & 2.3679 & 2.3685 & 2.3685\\
\bottomrule
\end{tabular}
}
    \caption{Valores obtidos para o item $a$ com $h=0.1$}
\end{table}

\begin{table}[H]
    \centering
    \resizebox{\textwidth}{!}{
\begin{tabular}{ccccc}
\toprule
x & Erro absoluto (Euler Modificado) & Erro relativo (Euler Modificado) & Erro absoluto (Euler Aperfeiçoado) & Erro relativo (Euler Aperfeiçoado)\\
\midrule
0.00e+00 & 0.00e+00 & 0.00e+00 & 0.00e+00 & 0.00e+00\\
1.00e-01 & 1.63e-04 & 8.11e-05 & 1.63e-04 & 8.11e-05\\
2.00e-01 & 2.94e-04 & 1.46e-04 & 2.94e-04 & 1.46e-04\\
3.00e-01 & 3.99e-04 & 1.96e-04 & 3.99e-04 & 1.96e-04\\
4.00e-01 & 4.82e-04 & 2.33e-04 & 4.82e-04 & 2.33e-04\\
5.00e-01 & 5.45e-04 & 2.59e-04 & 5.45e-04 & 2.59e-04\\
6.00e-01 & 5.92e-04 & 2.75e-04 & 5.92e-04 & 2.75e-04\\
7.00e-01 & 6.25e-04 & 2.84e-04 & 6.25e-04 & 2.84e-04\\
8.00e-01 & 6.46e-04 & 2.87e-04 & 6.46e-04 & 2.87e-04\\
9.00e-01 & 6.58e-04 & 2.85e-04 & 6.58e-04 & 2.85e-04\\
1.00e+00 & 6.62e-04 & 2.79e-04 & 6.62e-04 & 2.79e-04\\
\bottomrule
\end{tabular}
}
    \caption{Erros obtidos para o item $a$ com $h=0.1$}
\end{table}

\begin{figure}[H]
    \includegraphics[width=\linewidth]{results/ex3/a/h_0.1.png}
    \caption{Gráfico plotado para o item $a$ quando $h=0.1$}
\end{figure}

\begin{figure}[H]
    \includegraphics[width=\linewidth]{results/ex3/a/h_0.1_abs_error.png}
    \caption{Gráfico plotado para o erro absoluto do item $a$ quando $h=0.1$}
\end{figure}

\begin{figure}[H]
    \includegraphics[width=\linewidth]{results/ex3/a/h_0.1_rel_error.png}
    \caption{Gráfico plotado para o erro relativo do item $a$ quando $h=0.1$}
\end{figure}
\subsubsection{$h=0.01$}

\begin{table}[H]
    \centering
    \resizebox{\textwidth}{!}{
\begin{tabular}{cccc}
\toprule
x & Valor real & Valor (Euler Modificado) & Valor (Euler Aperfeiçoado)\\
\midrule
0.0000 & 2.0000 & 2.0000 & 2.0000\\
0.1000 & 2.0048 & 2.0048 & 2.0048\\
0.2000 & 2.0187 & 2.0187 & 2.0187\\
0.3000 & 2.0408 & 2.0408 & 2.0408\\
0.4000 & 2.0703 & 2.0703 & 2.0703\\
0.5000 & 2.1065 & 2.1065 & 2.1065\\
0.6000 & 2.1488 & 2.1488 & 2.1488\\
0.7000 & 2.1966 & 2.1966 & 2.1966\\
0.8000 & 2.2493 & 2.2493 & 2.2493\\
0.9000 & 2.3066 & 2.3066 & 2.3066\\
1.0000 & 2.3679 & 2.3679 & 2.3679\\
\bottomrule
\end{tabular}
}
    \caption{Valores obtidos para o item $a$ com $h=0.01$}
\end{table}

\begin{table}[H]
    \centering
    \resizebox{\textwidth}{!}{
\begin{tabular}{ccccc}
\toprule
x & Erro absoluto (Euler Modificado) & Erro relativo (Euler Modificado) & Erro absoluto (Euler Aperfeiçoado) & Erro relativo (Euler Aperfeiçoado)\\
\midrule
0.00e+00 & 0.00e+00 & 0.00e+00 & 0.00e+00 & 0.00e+00\\
1.00e-01 & 1.52e-06 & 7.58e-07 & 1.52e-06 & 7.58e-07\\
2.00e-01 & 2.75e-06 & 1.36e-06 & 2.75e-06 & 1.36e-06\\
3.00e-01 & 3.73e-06 & 1.83e-06 & 3.73e-06 & 1.83e-06\\
4.00e-01 & 4.50e-06 & 2.17e-06 & 4.50e-06 & 2.17e-06\\
5.00e-01 & 5.09e-06 & 2.42e-06 & 5.09e-06 & 2.42e-06\\
6.00e-01 & 5.53e-06 & 2.57e-06 & 5.53e-06 & 2.57e-06\\
7.00e-01 & 5.84e-06 & 2.66e-06 & 5.84e-06 & 2.66e-06\\
8.00e-01 & 6.04e-06 & 2.68e-06 & 6.04e-06 & 2.68e-06\\
9.00e-01 & 6.14e-06 & 2.66e-06 & 6.14e-06 & 2.66e-06\\
1.00e+00 & 6.18e-06 & 2.61e-06 & 6.18e-06 & 2.61e-06\\
\bottomrule
\end{tabular}
}
    \caption{Erros obtidos para o item $a$ com $h=0.01$}
\end{table}

\begin{figure}[H]
    \includegraphics[width=\linewidth]{results/ex3/a/h_0.01.png}
    \caption{Gráfico plotado para o item $a$ quando $h=0.01$}
\end{figure}

\begin{figure}[H]
    \includegraphics[width=\linewidth]{results/ex3/a/h_0.01_abs_error.png}
    \caption{Gráfico plotado para o erro absoluto do item $a$ quando $h=0.01$}
\end{figure}

\begin{figure}[H]
    \includegraphics[width=\linewidth]{results/ex3/a/h_0.01_rel_error.png}
    \caption{Gráfico plotado para o erro relativo do item $a$ quando $h=0.01$}
\end{figure}
\subsubsection{$h=0.005$}

\begin{table}[H]
    \centering
    \resizebox{\textwidth}{!}{
\begin{tabular}{cccc}
\toprule
x & Valor real & Valor (Euler Modificado) & Valor (Euler Aperfeiçoado)\\
\midrule
0.0000 & 2.0000 & 2.0000 & 2.0000\\
0.1000 & 2.0048 & 2.0048 & 2.0048\\
0.2000 & 2.0187 & 2.0187 & 2.0187\\
0.3000 & 2.0408 & 2.0408 & 2.0408\\
0.4000 & 2.0703 & 2.0703 & 2.0703\\
0.5000 & 2.1065 & 2.1065 & 2.1065\\
0.6000 & 2.1488 & 2.1488 & 2.1488\\
0.7000 & 2.1966 & 2.1966 & 2.1966\\
0.8000 & 2.2493 & 2.2493 & 2.2493\\
0.9000 & 2.3066 & 2.3066 & 2.3066\\
1.0000 & 2.3679 & 2.3679 & 2.3679\\
\bottomrule
\end{tabular}
}
    \caption{Valores obtidos para o item $a$ com $h=0.005$}
\end{table}

\begin{table}[H]
    \centering
    \resizebox{\textwidth}{!}{
\begin{tabular}{ccccc}
\toprule
x & Erro absoluto (Euler Modificado) & Erro relativo (Euler Modificado) & Erro absoluto (Euler Aperfeiçoado) & Erro relativo (Euler Aperfeiçoado)\\
\midrule
0.00e+00 & 0.00e+00 & 0.00e+00 & 0.00e+00 & 0.00e+00\\
1.00e-01 & 3.78e-07 & 1.89e-07 & 3.78e-07 & 1.89e-07\\
2.00e-01 & 6.85e-07 & 3.39e-07 & 6.85e-07 & 3.39e-07\\
3.00e-01 & 9.30e-07 & 4.55e-07 & 9.30e-07 & 4.55e-07\\
4.00e-01 & 1.12e-06 & 5.42e-07 & 1.12e-06 & 5.42e-07\\
5.00e-01 & 1.27e-06 & 6.02e-07 & 1.27e-06 & 6.02e-07\\
6.00e-01 & 1.38e-06 & 6.41e-07 & 1.38e-06 & 6.41e-07\\
7.00e-01 & 1.45e-06 & 6.62e-07 & 1.45e-06 & 6.62e-07\\
8.00e-01 & 1.50e-06 & 6.68e-07 & 1.50e-06 & 6.68e-07\\
9.00e-01 & 1.53e-06 & 6.63e-07 & 1.53e-06 & 6.63e-07\\
1.00e+00 & 1.54e-06 & 6.50e-07 & 1.54e-06 & 6.50e-07\\
\bottomrule
\end{tabular}
}
    \caption{Erros obtidos para o item $a$ com $h=0.005$}
\end{table}

\begin{figure}[H]
    \includegraphics[width=\linewidth]{results/ex3/a/h_0.005.png}
    \caption{Gráfico plotado para o item $a$ quando $h=0.005$}
\end{figure}

\begin{figure}[H]
    \includegraphics[width=\linewidth]{results/ex3/a/h_0.005_abs_error.png}
    \caption{Gráfico plotado para o erro absoluto do item $a$ quando $h=0.005$}
\end{figure}

\begin{figure}[H]
    \includegraphics[width=\linewidth]{results/ex3/a/h_0.005_rel_error.png}
    \caption{Gráfico plotado para o erro relativo do item $a$ quando $h=0.005$}
\end{figure}
\subsubsection{$h=0.001$}

\begin{table}[H]
    \centering
    \resizebox{\textwidth}{!}{
\begin{tabular}{cccc}
\toprule
x & Valor real & Valor (Euler Modificado) & Valor (Euler Aperfeiçoado)\\
\midrule
0.0000 & 2.0000 & 2.0000 & 2.0000\\
0.1000 & 2.0048 & 2.0048 & 2.0048\\
0.2000 & 2.0187 & 2.0187 & 2.0187\\
0.3000 & 2.0408 & 2.0408 & 2.0408\\
0.4000 & 2.0703 & 2.0703 & 2.0703\\
0.5000 & 2.1065 & 2.1065 & 2.1065\\
0.6000 & 2.1488 & 2.1488 & 2.1488\\
0.7000 & 2.1966 & 2.1966 & 2.1966\\
0.8000 & 2.2493 & 2.2493 & 2.2493\\
0.9000 & 2.3066 & 2.3066 & 2.3066\\
1.0000 & 2.3679 & 2.3679 & 2.3679\\
\bottomrule
\end{tabular}
}
    \caption{Valores obtidos para o item $a$ com $h=0.001$}
\end{table}

\begin{table}[H]
    \centering
    \resizebox{\textwidth}{!}{
\begin{tabular}{ccccc}
\toprule
x & Erro absoluto (Euler Modificado) & Erro relativo (Euler Modificado) & Erro absoluto (Euler Aperfeiçoado) & Erro relativo (Euler Aperfeiçoado)\\
\midrule
0.00e+00 & 0.00e+00 & 0.00e+00 & 0.00e+00 & 0.00e+00\\
1.00e-01 & 1.51e-08 & 7.53e-09 & 1.51e-08 & 7.53e-09\\
2.00e-01 & 2.73e-08 & 1.35e-08 & 2.73e-08 & 1.35e-08\\
3.00e-01 & 3.71e-08 & 1.82e-08 & 3.71e-08 & 1.82e-08\\
4.00e-01 & 4.47e-08 & 2.16e-08 & 4.47e-08 & 2.16e-08\\
5.00e-01 & 5.06e-08 & 2.40e-08 & 5.06e-08 & 2.40e-08\\
6.00e-01 & 5.49e-08 & 2.56e-08 & 5.49e-08 & 2.56e-08\\
7.00e-01 & 5.80e-08 & 2.64e-08 & 5.80e-08 & 2.64e-08\\
8.00e-01 & 6.00e-08 & 2.67e-08 & 6.00e-08 & 2.67e-08\\
9.00e-01 & 6.10e-08 & 2.65e-08 & 6.10e-08 & 2.65e-08\\
1.00e+00 & 6.14e-08 & 2.59e-08 & 6.14e-08 & 2.59e-08\\
\bottomrule
\end{tabular}
}
    \caption{Erros obtidos para o item $a$ com $h=0.001$}
\end{table}

\begin{figure}[H]
    \includegraphics[width=\linewidth]{results/ex3/a/h_0.001.png}
    \caption{Gráfico plotado para o item $a$ quando $h=0.001$}
\end{figure}

\begin{figure}[H]
    \includegraphics[width=\linewidth]{results/ex3/a/h_0.001_abs_error.png}
    \caption{Gráfico plotado para o erro absoluto do item $a$ quando $h=0.001$}
\end{figure}

\begin{figure}[H]
    \includegraphics[width=\linewidth]{results/ex3/a/h_0.001_rel_error.png}
    \caption{Gráfico plotado para o erro relativo do item $a$ quando $h=0.001$}
\end{figure}
\subsection{Exercício 3, item $b$}\subsubsection{$h=0.1$}

\begin{table}[H]
    \centering
    \resizebox{\textwidth}{!}{
\begin{tabular}{cccc}
\toprule
x & Valor real & Valor (Euler Modificado) & Valor (Euler Aperfeiçoado)\\
\midrule
0.0000 & 0.5000 & 0.5000 & 0.5000\\
0.1000 & 0.4975 & 0.4975 & 0.4975\\
0.2000 & 0.4902 & 0.4901 & 0.4902\\
0.3000 & 0.4785 & 0.4784 & 0.4784\\
0.4000 & 0.4630 & 0.4628 & 0.4629\\
0.5000 & 0.4444 & 0.4442 & 0.4444\\
0.6000 & 0.4237 & 0.4235 & 0.4237\\
0.7000 & 0.4016 & 0.4013 & 0.4016\\
0.8000 & 0.3788 & 0.3785 & 0.3788\\
0.9000 & 0.3559 & 0.3556 & 0.3560\\
1.0000 & 0.3333 & 0.3331 & 0.3335\\
\bottomrule
\end{tabular}
}
    \caption{Valores obtidos para o item $b$ com $h=0.1$}
\end{table}

\begin{table}[H]
    \centering
    \resizebox{\textwidth}{!}{
\begin{tabular}{ccccc}
\toprule
x & Erro absoluto (Euler Modificado) & Erro relativo (Euler Modificado) & Erro absoluto (Euler Aperfeiçoado) & Erro relativo (Euler Aperfeiçoado)\\
\midrule
0.00e+00 & 0.00e+00 & 0.00e+00 & 0.00e+00 & 0.00e+00\\
1.00e-01 & 1.24e-05 & 2.50e-05 & 1.24e-05 & 2.50e-05\\
2.00e-01 & 4.76e-05 & 9.70e-05 & 2.32e-05 & 4.74e-05\\
3.00e-01 & 9.83e-05 & 2.05e-04 & 2.97e-05 & 6.20e-05\\
4.00e-01 & 1.55e-04 & 3.34e-04 & 2.89e-05 & 6.24e-05\\
5.00e-01 & 2.06e-04 & 4.64e-04 & 1.91e-05 & 4.29e-05\\
6.00e-01 & 2.45e-04 & 5.79e-04 & 2.60e-08 & 6.13e-08\\
7.00e-01 & 2.67e-04 & 6.65e-04 & 2.70e-05 & 6.72e-05\\
8.00e-01 & 2.71e-04 & 7.15e-04 & 5.96e-05 & 1.57e-04\\
9.00e-01 & 2.59e-04 & 7.28e-04 & 9.48e-05 & 2.66e-04\\
1.00e+00 & 2.35e-04 & 7.04e-04 & 1.30e-04 & 3.89e-04\\
\bottomrule
\end{tabular}
}
    \caption{Erros obtidos para o item $b$ com $h=0.1$}
\end{table}

\begin{figure}[H]
    \includegraphics[width=\linewidth]{results/ex3/b/h_0.1.png}
    \caption{Gráfico plotado para o item $b$ quando $h=0.1$}
\end{figure}

\begin{figure}[H]
    \includegraphics[width=\linewidth]{results/ex3/b/h_0.1_abs_error.png}
    \caption{Gráfico plotado para o erro absoluto do item $b$ quando $h=0.1$}
\end{figure}

\begin{figure}[H]
    \includegraphics[width=\linewidth]{results/ex3/b/h_0.1_rel_error.png}
    \caption{Gráfico plotado para o erro relativo do item $b$ quando $h=0.1$}
\end{figure}
\subsubsection{$h=0.01$}

\begin{table}[H]
    \centering
    \resizebox{\textwidth}{!}{
\begin{tabular}{cccc}
\toprule
x & Valor real & Valor (Euler Modificado) & Valor (Euler Aperfeiçoado)\\
\midrule
0.0000 & 0.5000 & 0.5000 & 0.5000\\
0.1000 & 0.4975 & 0.4975 & 0.4975\\
0.2000 & 0.4902 & 0.4902 & 0.4902\\
0.3000 & 0.4785 & 0.4785 & 0.4785\\
0.4000 & 0.4630 & 0.4630 & 0.4630\\
0.5000 & 0.4444 & 0.4444 & 0.4444\\
0.6000 & 0.4237 & 0.4237 & 0.4237\\
0.7000 & 0.4016 & 0.4016 & 0.4016\\
0.8000 & 0.3788 & 0.3788 & 0.3788\\
0.9000 & 0.3559 & 0.3559 & 0.3559\\
1.0000 & 0.3333 & 0.3333 & 0.3333\\
\bottomrule
\end{tabular}
}
    \caption{Valores obtidos para o item $b$ com $h=0.01$}
\end{table}

\begin{table}[H]
    \centering
    \resizebox{\textwidth}{!}{
\begin{tabular}{ccccc}
\toprule
x & Erro absoluto (Euler Modificado) & Erro relativo (Euler Modificado) & Erro absoluto (Euler Aperfeiçoado) & Erro relativo (Euler Aperfeiçoado)\\
\midrule
0.00e+00 & 0.00e+00 & 0.00e+00 & 0.00e+00 & 0.00e+00\\
1.00e-01 & 1.23e-07 & 2.47e-07 & 1.17e-08 & 2.35e-08\\
2.00e-01 & 4.63e-07 & 9.45e-07 & 1.42e-08 & 2.90e-08\\
3.00e-01 & 9.46e-07 & 1.98e-06 & 1.08e-08 & 2.26e-08\\
4.00e-01 & 1.47e-06 & 3.18e-06 & 8.43e-08 & 1.82e-07\\
5.00e-01 & 1.94e-06 & 4.37e-06 & 2.20e-07 & 4.96e-07\\
6.00e-01 & 2.29e-06 & 5.40e-06 & 4.22e-07 & 9.96e-07\\
7.00e-01 & 2.47e-06 & 6.16e-06 & 6.79e-07 & 1.69e-06\\
8.00e-01 & 2.49e-06 & 6.57e-06 & 9.72e-07 & 2.57e-06\\
9.00e-01 & 2.36e-06 & 6.64e-06 & 1.28e-06 & 3.59e-06\\
1.00e+00 & 2.12e-06 & 6.37e-06 & 1.57e-06 & 4.72e-06\\
\bottomrule
\end{tabular}
}
    \caption{Erros obtidos para o item $b$ com $h=0.01$}
\end{table}

\begin{figure}[H]
    \includegraphics[width=\linewidth]{results/ex3/b/h_0.01.png}
    \caption{Gráfico plotado para o item $b$ quando $h=0.01$}
\end{figure}

\begin{figure}[H]
    \includegraphics[width=\linewidth]{results/ex3/b/h_0.01_abs_error.png}
    \caption{Gráfico plotado para o erro absoluto do item $b$ quando $h=0.01$}
\end{figure}

\begin{figure}[H]
    \includegraphics[width=\linewidth]{results/ex3/b/h_0.01_rel_error.png}
    \caption{Gráfico plotado para o erro relativo do item $b$ quando $h=0.01$}
\end{figure}
\subsubsection{$h=0.005$}

\begin{table}[H]
    \centering
    \resizebox{\textwidth}{!}{
\begin{tabular}{cccc}
\toprule
x & Valor real & Valor (Euler Modificado) & Valor (Euler Aperfeiçoado)\\
\midrule
0.0000 & 0.5000 & 0.5000 & 0.5000\\
0.1000 & 0.4975 & 0.4975 & 0.4975\\
0.2000 & 0.4902 & 0.4902 & 0.4902\\
0.3000 & 0.4785 & 0.4785 & 0.4785\\
0.4000 & 0.4630 & 0.4630 & 0.4630\\
0.5000 & 0.4444 & 0.4444 & 0.4444\\
0.6000 & 0.4237 & 0.4237 & 0.4237\\
0.7000 & 0.4016 & 0.4016 & 0.4016\\
0.8000 & 0.3788 & 0.3788 & 0.3788\\
0.9000 & 0.3559 & 0.3559 & 0.3559\\
1.0000 & 0.3333 & 0.3333 & 0.3333\\
\bottomrule
\end{tabular}
}
    \caption{Valores obtidos para o item $b$ com $h=0.005$}
\end{table}

\begin{table}[H]
    \centering
    \resizebox{\textwidth}{!}{
\begin{tabular}{ccccc}
\toprule
x & Erro absoluto (Euler Modificado) & Erro relativo (Euler Modificado) & Erro absoluto (Euler Aperfeiçoado) & Erro relativo (Euler Aperfeiçoado)\\
\midrule
0.00e+00 & 0.00e+00 & 0.00e+00 & 0.00e+00 & 0.00e+00\\
1.00e-01 & 3.07e-08 & 6.16e-08 & 1.39e-09 & 2.79e-09\\
2.00e-01 & 1.16e-07 & 2.36e-07 & 6.06e-10 & 1.24e-09\\
3.00e-01 & 2.36e-07 & 4.93e-07 & 6.83e-09 & 1.43e-08\\
4.00e-01 & 3.67e-07 & 7.93e-07 & 2.60e-08 & 5.62e-08\\
5.00e-01 & 4.84e-07 & 1.09e-06 & 6.05e-08 & 1.36e-07\\
6.00e-01 & 5.70e-07 & 1.35e-06 & 1.11e-07 & 2.62e-07\\
7.00e-01 & 6.16e-07 & 1.53e-06 & 1.75e-07 & 4.36e-07\\
8.00e-01 & 6.20e-07 & 1.64e-06 & 2.48e-07 & 6.54e-07\\
9.00e-01 & 5.88e-07 & 1.65e-06 & 3.24e-07 & 9.10e-07\\
1.00e+00 & 5.28e-07 & 1.58e-06 & 3.97e-07 & 1.19e-06\\
\bottomrule
\end{tabular}
}
    \caption{Erros obtidos para o item $b$ com $h=0.005$}
\end{table}

\begin{figure}[H]
    \includegraphics[width=\linewidth]{results/ex3/b/h_0.005.png}
    \caption{Gráfico plotado para o item $b$ quando $h=0.005$}
\end{figure}

\begin{figure}[H]
    \includegraphics[width=\linewidth]{results/ex3/b/h_0.005_abs_error.png}
    \caption{Gráfico plotado para o erro absoluto do item $b$ quando $h=0.005$}
\end{figure}

\begin{figure}[H]
    \includegraphics[width=\linewidth]{results/ex3/b/h_0.005_rel_error.png}
    \caption{Gráfico plotado para o erro relativo do item $b$ quando $h=0.005$}
\end{figure}
\subsubsection{$h=0.001$}

\begin{table}[H]
    \centering
    \resizebox{\textwidth}{!}{
\begin{tabular}{cccc}
\toprule
x & Valor real & Valor (Euler Modificado) & Valor (Euler Aperfeiçoado)\\
\midrule
0.0000 & 0.5000 & 0.5000 & 0.5000\\
0.1000 & 0.4975 & 0.4975 & 0.4975\\
0.2000 & 0.4902 & 0.4902 & 0.4902\\
0.3000 & 0.4785 & 0.4785 & 0.4785\\
0.4000 & 0.4630 & 0.4630 & 0.4630\\
0.5000 & 0.4444 & 0.4444 & 0.4444\\
0.6000 & 0.4237 & 0.4237 & 0.4237\\
0.7000 & 0.4016 & 0.4016 & 0.4016\\
0.8000 & 0.3788 & 0.3788 & 0.3788\\
0.9000 & 0.3559 & 0.3559 & 0.3559\\
1.0000 & 0.3333 & 0.3333 & 0.3333\\
\bottomrule
\end{tabular}
}
    \caption{Valores obtidos para o item $b$ com $h=0.001$}
\end{table}

\begin{table}[H]
    \centering
    \resizebox{\textwidth}{!}{
\begin{tabular}{ccccc}
\toprule
x & Erro absoluto (Euler Modificado) & Erro relativo (Euler Modificado) & Erro absoluto (Euler Aperfeiçoado) & Erro relativo (Euler Aperfeiçoado)\\
\midrule
0.00e+00 & 0.00e+00 & 0.00e+00 & 0.00e+00 & 0.00e+00\\
1.00e-01 & 1.23e-09 & 2.46e-09 & 6.17e-12 & 1.24e-11\\
2.00e-01 & 4.62e-09 & 9.42e-09 & 7.01e-11 & 1.43e-10\\
3.00e-01 & 9.42e-09 & 1.97e-08 & 4.04e-10 & 8.45e-10\\
4.00e-01 & 1.46e-08 & 3.16e-08 & 1.20e-09 & 2.59e-09\\
5.00e-01 & 1.93e-08 & 4.35e-08 & 2.59e-09 & 5.83e-09\\
6.00e-01 & 2.27e-08 & 5.37e-08 & 4.61e-09 & 1.09e-08\\
7.00e-01 & 2.46e-08 & 6.11e-08 & 7.17e-09 & 1.78e-08\\
8.00e-01 & 2.47e-08 & 6.52e-08 & 1.01e-08 & 2.66e-08\\
9.00e-01 & 2.34e-08 & 6.58e-08 & 1.31e-08 & 3.68e-08\\
1.00e+00 & 2.10e-08 & 6.31e-08 & 1.60e-08 & 4.80e-08\\
\bottomrule
\end{tabular}
}
    \caption{Erros obtidos para o item $b$ com $h=0.001$}
\end{table}

\begin{figure}[H]
    \includegraphics[width=\linewidth]{results/ex3/b/h_0.001.png}
    \caption{Gráfico plotado para o item $b$ quando $h=0.001$}
\end{figure}

\begin{figure}[H]
    \includegraphics[width=\linewidth]{results/ex3/b/h_0.001_abs_error.png}
    \caption{Gráfico plotado para o erro absoluto do item $b$ quando $h=0.001$}
\end{figure}

\begin{figure}[H]
    \includegraphics[width=\linewidth]{results/ex3/b/h_0.001_rel_error.png}
    \caption{Gráfico plotado para o erro relativo do item $b$ quando $h=0.001$}
\end{figure}

Nota-se que apesar de obter erros parecidos, os mesmos são quase sempre pelo menos uma ordem de grandeza superiores aos erros obtidos pelos métodos Runge-Kutta de Terceira e Quarta ordem. Isso é esperado devido a maior ordem dos métodos, pois os métodos de Euler Modificado e Euler Aperfeiçoado são de segunda ordem.

\end{document}
