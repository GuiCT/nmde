\documentclass[12pt]{article}
\usepackage[portuguese]{babel}
\usepackage[a4paper, left=30mm, right=20mm, top=30mm, bottom=20mm]{geometry}
\usepackage{setspace}
\usepackage{graphicx}
\usepackage{indentfirst}
\usepackage{amsmath}
\usepackage{siunitx}
\usepackage{csvsimple}
\usepackage{float}

% Parâmetros da Capa
\newcommand{\titulo}{1ª Tarefa de Métodos Computacionais para Equações Diferenciais}
\newcommand{\autor}{Guilherme Cesar Tomiasi}
\newcommand{\cidade}{Presidente Prudente}
% ================================================================================

\begin{document}
%Definir o espaçamento entre as linhas para 1.5
\setstretch{1.5}

%Capa------------------------------------------------------
\thispagestyle{empty}

\begin{minipage}[c]{0.3\textwidth}
\includegraphics[width=\textwidth]{../images/unesp.png}
\end{minipage}
\hspace{10pt}
\begin{minipage}[c]{0.6\textwidth}
\textbf{\uppercase{Universidade Estadual Paulista \\``Júlio de Mesquita Filho"}}
\end{minipage}

\setstretch{1.0}
\begin{center}
    \noindent\hrulefill

    \textbf{FCT - Faculdade de Ciências e Tecnologia}

    \textbf{DMC - Departamento de Matemática e Computação}

    \textbf{Pós-Graduação em Matemática Aplicada e Computacional}
\end{center}

\vspace*{\fill}

\setstretch{1.5}
\begin{center}
    \titulo\\
    \autor
\end{center}

\vspace{1cm}

\vspace*{\fill}

\begin{center}
    \uppercase{\cidade\\\today}
\end{center}
\singlespacing
\newpage
\tableofcontents

\newpage
\section{Descrição da Tarefa}
São oferecidas quatro funções, as quais precisamos aproximar as derivadas em um determinado intervalo. Pede-se para realizar a aproximação de cada uma utilizando os métodos de diferenças finitas avançada, atrasada e centrada. Todas as funções apresentadas possuem solução analítica, de forma que é possível saber o erro de cada método. Com isso em mente, pede-se para avaliar o comportamento de cada método com diferentes passos (representado pela letra $h$).

Tendo em vista que os métodos não mudam de uma função para outra, e que todos os cálculos são feitos de forma explícita, a última etapa para automatizar esse processo é fazer a leitura da função cuja derivada será aproximada (esse processo é chamado de \textit{parsing} na programação). A Linguagem de Programação utilizada (Julia) permite fazer a instanciação dessas funções a partir de uma entrada de texto. Portanto, para realizar a atividade, foi criado um arquivo JSON contendo cada uma das funções e intervalos pedidos.

Esse arquivo é lido por um \textit{script}, que realiza a leitura, instanciação e aproximação de cada uma das funções no intervalo pedido, para os seguintes passos de malha:
\begin{enumerate}
    \item $h=0.1$ ou $h=1 \text{e-}1$;
    \item $h=0.001$ ou $h=1 \text{e-}3$;
    \item $h=0.00001$ ou $h=1 \text{e-}5$;
\end{enumerate}
Esses valores foram escolhidos pois permitem visualizar de forma intuitiva o aumento da precisão dos métodos. Caso utilizássemos valores menores, não seria possível ver muita diferença. Os resultados para cada malha são salvos em arquivos tabulados (CSV), e as representações via gráficos são salvas em arquivos de imagem (PNG).

\section{Resultados}
Como as tabelas possuem números muito longos em precisão, as mesmas estão muito pequenas. No entanto, o \textit{zoom} pode ser utilizado para permitir a leitura.

\subsection{Item $a$}
\subsubsection{$h=0.1$}
\begin{table}[H]
    \centering
    \resizebox{\textwidth}{!}{
        \begin{tabular}{ll|lllllllll}
        \hline
        \csvreader[
            column count=11,
            no head,
            table head=\hline,
            late after line=\\\hline
        ]
        {results/a_h_0.1.csv}
        {1=\x,2=\ValorReal,3=\ValorAvancada,4=\ErroAvancada,5=\ErroAvancadaRelativa,6=\ValorAtrasada,7=\ErroAtrasada,8=\ErroAtrasadaRelativa,9=\ValorCentrada,10=\ErroCentrada,11=\ErroCentradaRelativa}
        {\x&\ValorReal&\ValorAvancada&\ErroAvancada&\ErroAvancadaRelativa&\ValorAtrasada&\ErroAtrasada&\ErroAtrasadaRelativa&\ValorCentrada&\ErroCentrada&\ErroCentradaRelativa}
    \end{tabular}}
    \caption{Item $a$, com $h=0.1$}
\end{table}

\begin{figure}[H]
    \includegraphics[width=\linewidth]{results/a_h_0.1.png}
    \caption{Gráfico plotado para o item $a$ quando $h=0.1$}
\end{figure}

\subsubsection{$h=0.01$}
\begin{table}[H]
    \centering
    \resizebox{\textwidth}{!}{
        \begin{tabular}{ll|lllllllll}
        \hline
        \csvreader[
            column count=11,
            no head,
            table head=\hline,
            late after line=\\\hline
        ]
        {results/a_h_0.01.csv}
        {1=\x,2=\ValorReal,3=\ValorAvancada,4=\ErroAvancada,5=\ErroAvancadaRelativa,6=\ValorAtrasada,7=\ErroAtrasada,8=\ErroAtrasadaRelativa,9=\ValorCentrada,10=\ErroCentrada,11=\ErroCentradaRelativa}
        {\x&\ValorReal&\ValorAvancada&\ErroAvancada&\ErroAvancadaRelativa&\ValorAtrasada&\ErroAtrasada&\ErroAtrasadaRelativa&\ValorCentrada&\ErroCentrada&\ErroCentradaRelativa}
    \end{tabular}}
    \caption{Item $a$, com $h=0.01$}
\end{table}

\begin{figure}[H]
    \includegraphics[width=\linewidth]{results/a_h_0.01.png}
    \caption{Gráfico plotado para o item $a$ quando $h=0.01$}
\end{figure}

\subsubsection{$h=0.0001$}
\begin{table}[H]
    \centering
    \resizebox{\textwidth}{!}{
        \begin{tabular}{ll|lllllllll}
        \hline
        \csvreader[
            column count=11,
            no head,
            table head=\hline,
            late after line=\\\hline
        ]
        {results/a_h_0.0001.csv}
        {1=\x,2=\ValorReal,3=\ValorAvancada,4=\ErroAvancada,5=\ErroAvancadaRelativa,6=\ValorAtrasada,7=\ErroAtrasada,8=\ErroAtrasadaRelativa,9=\ValorCentrada,10=\ErroCentrada,11=\ErroCentradaRelativa}
        {\x&\ValorReal&\ValorAvancada&\ErroAvancada&\ErroAvancadaRelativa&\ValorAtrasada&\ErroAtrasada&\ErroAtrasadaRelativa&\ValorCentrada&\ErroCentrada&\ErroCentradaRelativa}
    \end{tabular}}
    \caption{Item $a$, com $h=0.0001$}
\end{table}

\begin{figure}[H]
    \includegraphics[width=\linewidth]{results/a_h_0.0001.png}
    \caption{Gráfico plotado para o item $a$ quando $h=0.0001$}
\end{figure}


\subsection{Item $b$}
\subsubsection{$h=0.1$}
\begin{table}[H]
    \centering
    \resizebox{\textwidth}{!}{
        \begin{tabular}{ll|lllllllll}
        \hline
        \csvreader[
            column count=11,
            no head,
            table head=\hline,
            late after line=\\\hline
        ]
        {results/b_h_0.1.csv}
        {1=\x,2=\ValorReal,3=\ValorAvancada,4=\ErroAvancada,5=\ErroAvancadaRelativa,6=\ValorAtrasada,7=\ErroAtrasada,8=\ErroAtrasadaRelativa,9=\ValorCentrada,10=\ErroCentrada,11=\ErroCentradaRelativa}
        {\x&\ValorReal&\ValorAvancada&\ErroAvancada&\ErroAvancadaRelativa&\ValorAtrasada&\ErroAtrasada&\ErroAtrasadaRelativa&\ValorCentrada&\ErroCentrada&\ErroCentradaRelativa}
    \end{tabular}}
    \caption{Item $b$, com $h=0.1$}
\end{table}

\begin{figure}[H]
    \includegraphics[width=\linewidth]{results/b_h_0.1.png}
    \caption{Gráfico plotado para o item $b$ quando $h=0.1$}
\end{figure}

\subsubsection{$h=0.01$}
\begin{table}[H]
    \centering
    \resizebox{\textwidth}{!}{
        \begin{tabular}{ll|lllllllll}
        \hline
        \csvreader[
            column count=11,
            no head,
            table head=\hline,
            late after line=\\\hline
        ]
        {results/b_h_0.01.csv}
        {1=\x,2=\ValorReal,3=\ValorAvancada,4=\ErroAvancada,5=\ErroAvancadaRelativa,6=\ValorAtrasada,7=\ErroAtrasada,8=\ErroAtrasadaRelativa,9=\ValorCentrada,10=\ErroCentrada,11=\ErroCentradaRelativa}
        {\x&\ValorReal&\ValorAvancada&\ErroAvancada&\ErroAvancadaRelativa&\ValorAtrasada&\ErroAtrasada&\ErroAtrasadaRelativa&\ValorCentrada&\ErroCentrada&\ErroCentradaRelativa}
    \end{tabular}}
    \caption{Item $b$, com $h=0.01$}
\end{table}

\begin{figure}[H]
    \includegraphics[width=\linewidth]{results/b_h_0.01.png}
    \caption{Gráfico plotado para o item $b$ quando $h=0.01$}
\end{figure}

\subsubsection{$h=0.0001$}
\begin{table}[H]
    \centering
    \resizebox{\textwidth}{!}{
        \begin{tabular}{ll|lllllllll}
        \hline
        \csvreader[
            column count=11,
            no head,
            table head=\hline,
            late after line=\\\hline
        ]
        {results/b_h_0.0001.csv}
        {1=\x,2=\ValorReal,3=\ValorAvancada,4=\ErroAvancada,5=\ErroAvancadaRelativa,6=\ValorAtrasada,7=\ErroAtrasada,8=\ErroAtrasadaRelativa,9=\ValorCentrada,10=\ErroCentrada,11=\ErroCentradaRelativa}
        {\x&\ValorReal&\ValorAvancada&\ErroAvancada&\ErroAvancadaRelativa&\ValorAtrasada&\ErroAtrasada&\ErroAtrasadaRelativa&\ValorCentrada&\ErroCentrada&\ErroCentradaRelativa}
    \end{tabular}}
    \caption{Item $b$, com $h=0.0001$}
\end{table}

\begin{figure}[H]
    \includegraphics[width=\linewidth]{results/b_h_0.0001.png}
    \caption{Gráfico plotado para o item $b$ quando $h=0.0001$}
\end{figure}


\subsection{Item $c$}
\subsubsection{$h=0.1$}
\begin{table}[H]
    \centering
    \resizebox{\textwidth}{!}{
        \begin{tabular}{ll|lllllllll}
        \hline
        \csvreader[
            column count=11,
            no head,
            table head=\hline,
            late after line=\\\hline
        ]
        {results/c_h_0.1.csv}
        {1=\x,2=\ValorReal,3=\ValorAvancada,4=\ErroAvancada,5=\ErroAvancadaRelativa,6=\ValorAtrasada,7=\ErroAtrasada,8=\ErroAtrasadaRelativa,9=\ValorCentrada,10=\ErroCentrada,11=\ErroCentradaRelativa}
        {\x&\ValorReal&\ValorAvancada&\ErroAvancada&\ErroAvancadaRelativa&\ValorAtrasada&\ErroAtrasada&\ErroAtrasadaRelativa&\ValorCentrada&\ErroCentrada&\ErroCentradaRelativa}
    \end{tabular}}
    \caption{Item $c$, com $h=0.1$}
\end{table}

\begin{figure}[H]
    \includegraphics[width=\linewidth]{results/c_h_0.1.png}
    \caption{Gráfico plotado para o item $c$ quando $h=0.1$}
\end{figure}

\subsubsection{$h=0.01$}
\begin{table}[H]
    \centering
    \resizebox{\textwidth}{!}{
        \begin{tabular}{ll|lllllllll}
        \hline
        \csvreader[
            column count=11,
            no head,
            table head=\hline,
            late after line=\\\hline
        ]
        {results/c_h_0.01.csv}
        {1=\x,2=\ValorReal,3=\ValorAvancada,4=\ErroAvancada,5=\ErroAvancadaRelativa,6=\ValorAtrasada,7=\ErroAtrasada,8=\ErroAtrasadaRelativa,9=\ValorCentrada,10=\ErroCentrada,11=\ErroCentradaRelativa}
        {\x&\ValorReal&\ValorAvancada&\ErroAvancada&\ErroAvancadaRelativa&\ValorAtrasada&\ErroAtrasada&\ErroAtrasadaRelativa&\ValorCentrada&\ErroCentrada&\ErroCentradaRelativa}
    \end{tabular}}
    \caption{Item $c$, com $h=0.01$}
\end{table}

\begin{figure}[H]
    \includegraphics[width=\linewidth]{results/c_h_0.01.png}
    \caption{Gráfico plotado para o item $c$ quando $h=0.01$}
\end{figure}

\subsubsection{$h=0.0001$}
\begin{table}[H]
    \centering
    \resizebox{\textwidth}{!}{
        \begin{tabular}{ll|lllllllll}
        \hline
        \csvreader[
            column count=11,
            no head,
            table head=\hline,
            late after line=\\\hline
        ]
        {results/c_h_0.0001.csv}
        {1=\x,2=\ValorReal,3=\ValorAvancada,4=\ErroAvancada,5=\ErroAvancadaRelativa,6=\ValorAtrasada,7=\ErroAtrasada,8=\ErroAtrasadaRelativa,9=\ValorCentrada,10=\ErroCentrada,11=\ErroCentradaRelativa}
        {\x&\ValorReal&\ValorAvancada&\ErroAvancada&\ErroAvancadaRelativa&\ValorAtrasada&\ErroAtrasada&\ErroAtrasadaRelativa&\ValorCentrada&\ErroCentrada&\ErroCentradaRelativa}
    \end{tabular}}
    \caption{Item $c$, com $h=0.0001$}
\end{table}

\begin{figure}[H]
    \includegraphics[width=\linewidth]{results/c_h_0.0001.png}
    \caption{Gráfico plotado para o item $c$ quando $h=0.0001$}
\end{figure}


\subsection{Item $d$}
\subsubsection{$h=0.1$}
\begin{table}[H]
    \centering
    \resizebox{\textwidth}{!}{
        \begin{tabular}{ll|lllllllll}
        \hline
        \csvreader[
            column count=11,
            no head,
            table head=\hline,
            late after line=\\\hline
        ]
        {results/d_h_0.1.csv}
        {1=\x,2=\ValorReal,3=\ValorAvancada,4=\ErroAvancada,5=\ErroAvancadaRelativa,6=\ValorAtrasada,7=\ErroAtrasada,8=\ErroAtrasadaRelativa,9=\ValorCentrada,10=\ErroCentrada,11=\ErroCentradaRelativa}
        {\x&\ValorReal&\ValorAvancada&\ErroAvancada&\ErroAvancadaRelativa&\ValorAtrasada&\ErroAtrasada&\ErroAtrasadaRelativa&\ValorCentrada&\ErroCentrada&\ErroCentradaRelativa}
    \end{tabular}}
    \caption{Item $d$, com $h=0.1$}
\end{table}

\begin{figure}[H]
    \includegraphics[width=\linewidth]{results/d_h_0.1.png}
    \caption{Gráfico plotado para o item $d$ quando $h=0.1$}
\end{figure}

\subsubsection{$h=0.01$}
\begin{table}[H]
    \centering
    \resizebox{\textwidth}{!}{
        \begin{tabular}{ll|lllllllll}
        \hline
        \csvreader[
            column count=11,
            no head,
            table head=\hline,
            late after line=\\\hline
        ]
        {results/d_h_0.01.csv}
        {1=\x,2=\ValorReal,3=\ValorAvancada,4=\ErroAvancada,5=\ErroAvancadaRelativa,6=\ValorAtrasada,7=\ErroAtrasada,8=\ErroAtrasadaRelativa,9=\ValorCentrada,10=\ErroCentrada,11=\ErroCentradaRelativa}
        {\x&\ValorReal&\ValorAvancada&\ErroAvancada&\ErroAvancadaRelativa&\ValorAtrasada&\ErroAtrasada&\ErroAtrasadaRelativa&\ValorCentrada&\ErroCentrada&\ErroCentradaRelativa}
    \end{tabular}}
    \caption{Item $d$, com $h=0.01$}
\end{table}

\begin{figure}[H]
    \includegraphics[width=\linewidth]{results/d_h_0.01.png}
    \caption{Gráfico plotado para o item $d$ quando $h=0.01$}
\end{figure}

\subsubsection{$h=0.0001$}
\begin{table}[H]
    \centering
    \resizebox{\textwidth}{!}{
        \begin{tabular}{ll|lllllllll}
        \hline
        \csvreader[
            column count=11,
            no head,
            table head=\hline,
            late after line=\\\hline
        ]
        {results/d_h_0.0001.csv}
        {1=\x,2=\ValorReal,3=\ValorAvancada,4=\ErroAvancada,5=\ErroAvancadaRelativa,6=\ValorAtrasada,7=\ErroAtrasada,8=\ErroAtrasadaRelativa,9=\ValorCentrada,10=\ErroCentrada,11=\ErroCentradaRelativa}
        {\x&\ValorReal&\ValorAvancada&\ErroAvancada&\ErroAvancadaRelativa&\ValorAtrasada&\ErroAtrasada&\ErroAtrasadaRelativa&\ValorCentrada&\ErroCentrada&\ErroCentradaRelativa}
    \end{tabular}}
    \caption{Item $d$, com $h=0.0001$}
\end{table}

\begin{figure}[H]
    \includegraphics[width=\linewidth]{results/d_h_0.0001.png}
    \caption{Gráfico plotado para o item $d$ quando $h=0.0001$}
\end{figure}

\end{document}
