\documentclass[11pt]{beamer}
\usepackage[portuguese]{babel}
\usepackage{graphicx}
\usepackage{siunitx}
\usepackage{csvsimple}

\usetheme{Madrid}
\usecolortheme{default}
%Information to be included in the title page:
\title[T.1. -- MCED]{Tarefa 1 de Métodos Computacionais para Equações Diferenciais}
\author[Tomiasi, G.C.]{Guilherme Cesar Tomiasi\inst{1}}
\institute[UNESP] % (optional)
{
  \inst{1}
  Departamento de Matemática e Computação\\
  Universidade Estadual Paulista ``Júlio de Mesquita Filho''
}
\date{24 de março de 2025}

\AtBeginSection[]
{
  \begin{frame}
    \frametitle{Sumário}
    \tableofcontents[currentsection]
  \end{frame}
}

\begin{document}
\frame{\titlepage}

\section{Introdução ao problema}

\begin{frame}
    \frametitle{Aproximação via diferenças finitas}
    
    Dada uma determinada função $y$, é possível aproximar, numericamente, a sua derivada (dada por $y'$) a partir das seguintes fórmulas:
    \begin{itemize}
        \item Diferença \textbf{avançada}: $y'_x = \dfrac{y_{x+1} - y_x}{h} + O(h)$
        \item Diferença \textbf{atrasada}: $y'_x = \dfrac{y_{x} - y_{x-1}}{h} + O(h)$
        \item Diferença \textbf{centrada}: $y'_x = \dfrac{y_{x+1} - y_{x-1}}{2h} + O(h^2)$
    \end{itemize}
\end{frame}

\begin{frame}
    \frametitle{Comportamento esperado do errro}

    Observamos que as diferenças avançada e atrasada possuem ordem de erro dominada assintoticamente por $h$, enquanto a diferença centrada é dominada assintoticamente por $h^2$. Dessa maneira, é esperado que o erro do valor da derivada aproximado pela fórmula da diferença centrada seja muito menor para um mesmo valor de $h$.
\end{frame}

\section{Função escolhida para demonstração}

\begin{frame}
    Para fazer a demonstração desse comportamento, é necessário escolher uma função com solução analítica conhecida, calcular o erro de cada aproximação e observar a evolução desses valores. No caso demonstrado, foi escolhida a função $y(x)=\cos(x^2)$, cuja derivada analítica é $y'(x)=-2x\sin(x^2)$.
\end{frame}

\section{Resultados (gráficos)}

\begin{frame}
    \frametitle{$h=0.1$}
    \begin{figure}
        \includegraphics[width=0.8\linewidth]{results/customFunction_h_0.1.png}
        \caption{Resultados para $h=0.1$}
    \end{figure}
\end{frame}

\begin{frame}
    \frametitle{$h=0.01$}
    \begin{figure}
        \includegraphics[width=0.8\linewidth]{results/customFunction_h_0.01.png}
        \caption{Resultados para $h=0.01$}
    \end{figure}
\end{frame}

\begin{frame}
    \frametitle{$h=0.0001$}
    \begin{figure}
        \includegraphics[width=0.8\linewidth]{results/customFunction_h_0.0001.png}
        \caption{Resultados para $h=0.0001$}
    \end{figure}
\end{frame}

\section{Resultados (tabulados)}

\begin{frame}
    \frametitle{$h=0.1$}
    \begin{table}[H]
        \centering
        \resizebox{\textwidth}{!}{
            \begin{tabular}{ll|lllllllll}
            \hline
            \csvreader[
                column count=11,
                no head,
                table head=\hline,
                late after line=\\\hline
            ]
            {results/customFunction_h_0.1.csv}
            {1=\x,2=\ValorReal,3=\ValorAvancada,4=\ErroAvancada,5=\ErroAvancadaRelativa,6=\ValorAtrasada,7=\ErroAtrasada,8=\ErroAtrasadaRelativa,9=\ValorCentrada,10=\ErroCentrada,11=\ErroCentradaRelativa}
            {\x&\ValorReal&\ValorAvancada&\ErroAvancada&\ErroAvancadaRelativa&\ValorAtrasada&\ErroAtrasada&\ErroAtrasadaRelativa&\ValorCentrada&\ErroCentrada&\ErroCentradaRelativa}
        \end{tabular}}
        \caption{$h=0.1$}
    \end{table}
\end{frame}

\begin{frame}
    \frametitle{$h=0.01$}
    \begin{table}[H]
        \centering
        \resizebox{\textwidth}{!}{
            \begin{tabular}{ll|lllllllll}
            \hline
            \csvreader[
                column count=11,
                no head,
                table head=\hline,
                late after line=\\\hline
            ]
            {results/customFunction_h_0.01.csv}
            {1=\x,2=\ValorReal,3=\ValorAvancada,4=\ErroAvancada,5=\ErroAvancadaRelativa,6=\ValorAtrasada,7=\ErroAtrasada,8=\ErroAtrasadaRelativa,9=\ValorCentrada,10=\ErroCentrada,11=\ErroCentradaRelativa}
            {\x&\ValorReal&\ValorAvancada&\ErroAvancada&\ErroAvancadaRelativa&\ValorAtrasada&\ErroAtrasada&\ErroAtrasadaRelativa&\ValorCentrada&\ErroCentrada&\ErroCentradaRelativa}
        \end{tabular}}
        \caption{$h=0.01$}
    \end{table}
\end{frame}

\begin{frame}
    \frametitle{$h=0.0001$}
    \begin{table}[H]
        \centering
        \resizebox{\textwidth}{!}{
            \begin{tabular}{ll|lllllllll}
            \hline
            \csvreader[
                column count=11,
                no head,
                table head=\hline,
                late after line=\\\hline
            ]
            {results/customFunction_h_0.0001.csv}
            {1=\x,2=\ValorReal,3=\ValorAvancada,4=\ErroAvancada,5=\ErroAvancadaRelativa,6=\ValorAtrasada,7=\ErroAtrasada,8=\ErroAtrasadaRelativa,9=\ValorCentrada,10=\ErroCentrada,11=\ErroCentradaRelativa}
            {\x&\ValorReal&\ValorAvancada&\ErroAvancada&\ErroAvancadaRelativa&\ValorAtrasada&\ErroAtrasada&\ErroAtrasadaRelativa&\ValorCentrada&\ErroCentrada&\ErroCentradaRelativa}
        \end{tabular}}
        \caption{$h=0.0001$}
    \end{table}
\end{frame}

\section{Extra: Interatividade}

\end{document}